\chapter{The IceCube Neutrino Observatory}

\section{Neutrino interactions and detection}

\textbf{What interactions do neutrinos have?}
Neutrinos are the only neutral leptons in the standard model of particle physics, interacting only through the weak force and gravity.

\textbf{How do these interactions manifest at different energy scales?}
Although fundamentally neutrinos only interact through the exchange of weak bosons, there exist a wide range of processes that dominate the relevant physical behavior of neutrinos at different energy scales.
Such interactions include: nuclear capture, inverse beta-decay, quasi-elastic scattering, resonant particle production, coherent elastic scattering, deep inelastic scattering (DIS), and ultra-high energy interactions~\ref{Vannucci:2017rqs,Akimov:2017ade}.
Above $\si\TeV$ neutrino energies however, only two known processes remain relevant for detection: DIS and resonant $W$ production.
Deep inelastic scattering refers to processes that probe the fundamental components of hadrons.
For neutrinos, this means the exchange of a weak boson with a quark.
The momentum imparted to the quark will produce a hadronic cascade of secondary particles.
The details of the lepton side of the interaction depend heavily on the species of incident neutrino and weak boson exchanged.
We can divide these DIS interactions into two categories based on the weak boson exchanged.
Interactions involving the exchange of a $Z$ are referred to as ``neutral current'' (NC), and those exchanging a $W$ are referred to as ``charged current'' (CC).


\textbf{What are the products from neutrino interactions that allow us to detect neutrinos?}

\textbf{What is the process by which we detect these products in existing detectors?}


\section{The IceCube detector}

\begingroup
\graphicspath{{results/HESE_Final_Paper/}}
\subfile{results/HESE_Final_Paper/sections/detectorselection/detector}
\endgroup
