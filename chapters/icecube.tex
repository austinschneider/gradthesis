\chapter{The IceCube Neutrino Observatory}

\section{Neutrino interactions and detection}

\textbf{What interactions do neutrinos have?}
Neutrinos are the only neutral leptons in the standard model of particle physics, interacting only through the weak force and gravity.

\textbf{How do these interactions manifest at different energy scales?}
Although fundamentally neutrinos only interact through the exchange of weak bosons, there exist a wide range of processes that dominate the relevant physical behavior of neutrinos at different energy scales.
Such interactions include: nuclear capture, inverse beta-decay, quasi-elastic scattering, resonant particle production, coherent elastic scattering, deep inelastic scattering (DIS), and ultra-high energy interactions~\ref{Vannucci:2017rqs,Akimov:2017ade}.
Above $\si\TeV$ neutrino energies however, only two known processes remain relevant for detection: DIS and resonant $W$ production.
Deep inelastic scattering refers to processes that probe the fundamental components of hadrons.
For neutrinos, this means the exchange of a weak boson with a quark.
The momentum imparted to the quark will produce a hadronic cascade of secondary particles.
The details of the lepton side of the interaction depend heavily on the species of incident neutrino and weak boson exchanged.
We can divide these DIS interactions into two categories based on the weak boson exchanged.
Interactions involving the exchange of a $Z$ are referred to as ``neutral current'' (NC), and those exchanging a $W$ are referred to as ``charged current'' (CC).
Modern techniques for observing neutrino interactions rely on detecting the charged particle products of the initial interaction.
As a consequence of this, the observable energy can be very different for NC and CC events.
Both interactions produce a hadronic cascade, but on the leptonic side of the interaction NC event have an outgoing neutrino (not observable) while CC events have an outgoing charged lepton (observable).
These two interactions are shown in \reffig{fig:DIS}.

\begin{figure}
	\centering
	\begin{tikzpicture}
	\begin{feynman}
	\vertex (a);
	\vertex [below=of a] (b);
	\vertex [above left=of a] (c) {\(\nu_{l} / \overline \nu_{l}\)};
	\vertex [above right=of a] (d) {\(\nu_{l} / \overline \nu_{l}\)};
	\vertex [below left=of b] (e) {\(u/d\)};
	\vertex [below right=of b] (f) {\(u/d\)};
	\diagram* {
		(c) -- [fermion] (a),
		(a) -- [fermion] (d),
		(e) -- [fermion] (b),
		(b) -- [fermion] (f),
		(a) -- [boson, edge label=\(Z^0\)] (b),
	};
	\end{feynman}
	\end{tikzpicture}
	\begin{tikzpicture}
	\begin{feynman}
	\vertex (a);
	\vertex [below=of a] (b);
	\vertex [above left=of a] (c) {\(\nu_{l} / \overline \nu_{l}\)};
	\vertex [above right=of a] (d) {\(l^\pm\)};
	\vertex [below left=of b] (e) {\(u/d\)};
	\vertex [below right=of b] (f) {\(d/u\)};
	\diagram* {
		(c) -- [fermion] (a),
		(a) -- [fermion] (d),
		(e) -- [fermion] (b),
		(b) -- [fermion] (f),
		(a) -- [boson, edge label=\(W^\pm\)] (b),
	};
	\end{feynman}
	\end{tikzpicture}
	\caption{The neutrino deep inelastic scattering processes NC (left) and CC (right) in matter is shown in the figure above for interactions with nucleon component quarks.
	In both cases, significant momentum can be imparted to the outgoing quark which will result in the production of a hadronic particle cascade.
	In NC interactions, only the hadronic cascade may be detectable as the interaction product is a neutrino which is unlikely to undergo another interaction within the detection medium.
	Interactions of the CC variety on the other hand, produce a charged lepton in addition to the hadronic cascade.
	This charged lepton can also be detected if it receives enough energy.}
	\label{fig:DIS}
\end{figure}

\begin{figure}
\begin{tikzpicture}
\begin{feynman}
\vertex (a);
\vertex [right=of a] (b);
\vertex [above left=of a] (c) {\(\overline \nu_{e}\)};
\vertex [below left=of a] (d) {\(e^-\)};
\diagram* {
	(c) -- [fermion] (a),
	(a) -- [fermion] (d),
	(a) -- [boson, edge label=\(W^+\)] (b),
};
\end{feynman}
\end{tikzpicture}
\caption{}
\label{}
\end{figure}


\textbf{What are the products from neutrino interactions that allow us to detect neutrinos?}

\textbf{What is the process by which we detect these products in existing detectors?}


\section{The IceCube detector}

\begingroup
\graphicspath{{results/HESE_Final_Paper/}}
\subfile{results/HESE_Final_Paper/sections/detectorselection/detector}
\endgroup
