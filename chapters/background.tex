\chapter{Background Information}

\section{Neutrino interactions and detection}

\textbf{What interactions do neutrinos have?}
Neutrinos are the only neutral leptons in the standard model of particle physics, and interact only through the weak force and gravity.

\textbf{How do these interactions manifest at different energy scales?}
Although fundamentally neutrinos only interact through gravity and the exchange of weak bosons, there exist a wide range of processes that dominate the relevant physical behavior of neutrinos at different energy scales.
Such interactions include: nuclear capture, inverse beta-decay, quasi-elastic scattering, resonant particle production, coherent elastic scattering, deep inelastic scattering (DIS), and ultra-high energy interactions~\ref{Vannucci:2017rqs,Akimov:2017ade}.
Above $\si\TeV$ neutrino energies however, only two known processes remain relevant for detection: DIS and resonant $W$ production.
Deep inelastic scattering refers to processes that probe the fundamental components of hadrons.
For neutrinos, this means the exchange of a weak boson with a quark.
The momentum imparted to the quark will produce a hadronic cascade of secondary particles.
The details of the lepton side of the interaction depend heavily on the species of incident neutrino and weak boson exchanged.
We can divide these DIS interactions into two categories based on the weak boson exchanged.
Interactions involving the exchange of a $Z$ are referred to as ``neutral current'' (NC), and those exchanging a $W$ are referred to as ``charged current'' (CC).
Modern techniques for observing neutrino interactions rely on detecting the charged particle products of the initial interaction.
As a consequence of this, the observable energy can be very different for NC and CC events.
Both interactions produce a hadronic cascade, but on the leptonic side of the interaction NC event have an outgoing neutrino (not observable) while CC events have an outgoing charged lepton (observable).
These two interactions are shown in \reffig{fig:DIS}.

\begin{figure}
	\centering
	\begin{tikzpicture}
	\begin{feynman}
	\vertex (a);
	\vertex [below=of a] (b);
	\vertex [above left=of a] (c) {\(\nu_{l} / \overline \nu_{l}\)};
	\vertex [above right=of a] (d) {\(\nu_{l} / \overline \nu_{l}\)};
	\vertex [below left=of b] (e) {\(u/d\)};
	\vertex [below right=of b] (f) {\(u/d\)};
	\diagram* {
		(c) -- [fermion] (a),
		(a) -- [fermion] (d),
		(e) -- [fermion] (b),
		(b) -- [fermion] (f),
		(a) -- [boson, edge label=\(Z^0\)] (b),
	};
	\end{feynman}
	\end{tikzpicture}
	\begin{tikzpicture}
	\begin{feynman}
	\vertex (a);
	\vertex [below=of a] (b);
	\vertex [above left=of a] (c) {\(\nu_{l} / \overline \nu_{l}\)};
	\vertex [above right=of a] (d) {\(l^\pm\)};
	\vertex [below left=of b] (e) {\(u/d\)};
	\vertex [below right=of b] (f) {\(d/u\)};
	\diagram* {
		(c) -- [fermion] (a),
		(a) -- [fermion] (d),
		(e) -- [fermion] (b),
		(b) -- [fermion] (f),
		(a) -- [boson, edge label=\(W^\pm\)] (b),
	};
	\end{feynman}
	\end{tikzpicture}
	\caption{The neutrino deep inelastic scattering processes NC (left) and CC (right) in matter is shown in the figure above for interactions with nucleon component quarks.
	In both cases, significant momentum can be imparted to the outgoing quark which will result in the production of a hadronic particle cascade.
	In NC interactions, only the hadronic cascade may be detectable as the interaction product is a neutrino which is unlikely to undergo another interaction within the detection medium.
	Interactions of the CC variety on the other hand, produce a charged lepton in addition to the hadronic cascade.
	This charged lepton can also be detected if it receives enough energy.}
	\label{fig:DIS}
\end{figure}

The third interaction relevant above $\si\TeV$ neutrino energies is the resonant production of a $W$ boson, otherwise known as the Glashow resonance (GR)~\cite{Glashow:1960zz}.
In matter on Earth this process occurs when an anti-electron neutrino combines with an atomic electron to produce an on-shell $W^+$ as shown in~\reffig{fig:glashow}
If we consider the rest frame of the electron, then this resonance occurs at a neutrino energy of $\SI{6.3}\PeV$.
For atomic electrons we should consider the rest frame of the atom, and in this case there is a Doppler broadening of the resonance of $\sim\SI{20}\percent$ due to the orbital motion of the electrons~\cite{Loewy:2014zva}.
In practice this broadening is small in comparison to the energy resolution of modern neutrino detectors that have access to this energy scale.
The production of a $W^+$ and it's subsequent decay can result in either a hadronic shower similar to a NC interaction, or a leptonic final state similar to a CC interaction.
These two possibilities correspond to the hadronic and leptonic decay modes of the $W$ respectively.

\begin{figure}
	\centering
	\begin{tikzpicture}
	\begin{feynman}
	\vertex (a);
	\vertex [right=of a] (b);
	\vertex [above left=of a] (c) {\(\overline \nu_{e}\)};
	\vertex [below left=of a] (d) {\(e^-\)};
	\diagram* {
		(c) -- [fermion] (a),
		(a) -- [fermion] (d),
		(a) -- [boson, edge label=\(W^+\)] (b),
	};
	\end{feynman}
	\end{tikzpicture}
	\caption{The production of an on-shell $W^+$ boson through the combination of an anti-electron neutrino and electron.
	This resonant process occurs at neutrino energies around $\SI{6.3}\PeV$.}
	\label{fig:glashow}
\end{figure}

Through either a NC interaction or the hadronic decay of a $W$ boson, neutrinos can induce a hadronic shower.
In a hadronic shower, both charged hadrons and leptons are produced which can be detected through well established methods.
Charged current interactions produce a hadronic shower by imparting momentum to a quark which is then hadronized, although the charged lepton produced in the interaction is also detectable and can significantly alter the detection signature of the event.

Through either a CC interaction or the leptonic decay of a $W$ boson, neutrinos can produce a detectable charged lepton, although the detection signature differs depending on the flavor of charged lepton produced.
Focusing on dense detection media like ice or water, the detection signatures of the three flavor of charged lepton are as follows.
High energy electrons and positrons immediately interact with the detection media to initiate an electromagnetic cascade where charge leptons and high energy photons are alternately produced by one another.
This electromagnetic cascade develops in a roughly spherical fashion within the dense detection medium, but has a directional bias because of the momentum of the first charged lepton.

Muons from high energy neutrino interactions do not interact as readily as electrons and positrons due to their larger mass.
Instead, muons are able to travel several kilometers in dense media before eventually losing enough energy that they quickly decay.
Along their entire path length, muons lose energy by interacting with the detection medium.
These ``energy loss'' interactions include ionization, electron-positron pair production, bremsstrahlung, and photo-nuclear interactions.
Although these processes are highly stochastic, the average energy loss of muons approximately follows $-dE/dx=a+bE$ where $a$ is determined by the ionization energy losses and $b$ is defined by the other processes.
In general $a$ and $b$ and both functions of the muon energy $E$, but this linear approximation where $a$ and $b$ are constant holds locally as both are slowly varying as a function of $E$.
For muon energies above $\SI{1}\TeV$, the so called ``radiative'' term $bE$ dominates the average energy losses.
Thus, the energy loss rate in a detection medium can be used to infer the muon energy.
In practice the muon energy can be determined to within a factor of $2$, however improved techniques may achieve a resolution as small as $\SI{10}\percent$ in the future.

Taus produced similarly are also detectable.
The short decay length of a tau, $\SI{50}\meter / \si\PeV$, means it is likely to decay very close to the neutrino interaction vertex.
Taus decay hadronically with a branching ratio of $\SI{64.79}\percent$, which results in a hadronic shower.
Leptonic decay modes of the tau are decay to a charged lepton and corresponding neutrino of either electron or muon flavor.
In the electron case, an electromagnetic shower results; whereas in the muon case a far travelling muon is produced.
For taus produced via the decay of a $W$ boson, the event is indistinguishable from CC and NC interactions other than by resonance energy at which this process occurs, as tau energy losses are negligible in the short decay length.


\textbf{What are the products from neutrino interactions that allow us to detect neutrinos?}

\textbf{What is the process by which we detect these products in existing detectors?}


