\chapter{Conclusions and next steps\label{chapter:conclusions}}
\section{Analysis conclusions\label{sec:analysis_conclusions}}
\begingroup
\graphicspath{{results/HESE_Final_Paper/}}
\chapter{Conclusions and next steps\label{chapter:conclusions}}
\section{Analysis conclusions\label{sec:analysis_conclusions}}
\begingroup
\graphicspath{{results/HESE_Final_Paper/}}
\chapter{Conclusions and next steps\label{chapter:conclusions}}
\section{Analysis conclusions\label{sec:analysis_conclusions}}
\begingroup
\graphicspath{{results/HESE_Final_Paper/}}
\chapter{Conclusions and next steps\label{chapter:conclusions}}
\section{Analysis conclusions\label{sec:analysis_conclusions}}
\begingroup
\graphicspath{{results/HESE_Final_Paper/}}
\input{results/HESE_Final_Paper/sections/conclusions}
\endgroup
\FloatBarrier

\section{Next steps}
The analysis presented here updated previous studies of the HESE sample, and is only a small part of the picture for astophysical neutrinos and the mystery of cosmic ray origin.
However, the techniques and solutions developed here remain broadly applicable to future analyses, and will enable more precise measurements as more data is accumulated and additional samples are developed.

From the perspective of analyzing the astrophysical neutrino flux...
\endgroup
\FloatBarrier

\section{Next steps}
The analysis presented here updated previous studies of the HESE sample, and is only a small part of the picture for astophysical neutrinos and the mystery of cosmic ray origin.
However, the techniques and solutions developed here remain broadly applicable to future analyses, and will enable more precise measurements as more data is accumulated and additional samples are developed.

From the perspective of analyzing the astrophysical neutrino flux...
\endgroup
\FloatBarrier

\section{Next steps}
The analysis presented here updated previous studies of the HESE sample, and is only a small part of the picture for astophysical neutrinos and the mystery of cosmic ray origin.
However, the techniques and solutions developed here remain broadly applicable to future analyses, and will enable more precise measurements as more data is accumulated and additional samples are developed.

From the perspective of analyzing the astrophysical neutrino flux...
\endgroup
\FloatBarrier

\section{Next steps}
The analysis presented here updated previous studies of the HESE sample, and is only a small part of the picture for astophysical neutrinos and the mystery of cosmic ray origin.
However, the techniques and solutions developed here remain broadly applicable to future analyses, and will enable more precise measurements as more data is accumulated and additional samples are developed.

From the perspective of analyzing the astrophysical neutrino flux the next objectives are to expand the energy range and flavor information of the analysis, and to simultaneously examine these different channels for additional structure or differences between them.
Some analyses have already examined these channels, for instance the through-going muon neutrino sample which looks at muon neutrinos from the Northern hemisphere between $\SI{100}\GeV$ and $\SI{10}\PeV$ or the medium-energy starting event (MESE) sample which extends the HESE selection down to $\sim\SI{1}\TeV$.
However, these independent analyses have either lacked extended treatments of systematics, lacked enough data to make significant conclusions, or are too disparate for meaningful comparison to these results.
The goal then is to perform a simultaneous analysis of all these data samples with the decade or more of data collected so far, but in that analysis to also expand upon the techniques developed for this work.

\subsection{Data samples}
The medium-energy starting event (MESE) sample modifies the veto definitions of HESE and extends the selection to lower energies primarily with the use of a charge dependent fiducial volume cut.
The MESE selection has similar properties to HESE in that only starting events are selected, the sample consits mostly of cascades, and the there is a separation between veto and fiducial volume.
A similar set of calculations can be applied to estimate the backgrounds of the MESE sample, and the selection is susceptible to the same systematics.
Primarily the addition of this selection allows us to explore the lower energies of the astrophysical spectrum between $\SI{10}\TeV$ and $\SI{100}\TeV$.
The lower energy range comes with its own challenges though, mainly a much larger data sample and larger atmospheric backgrounds.
To ensure accurate measurements with this increased sample size and background we must ensure a higher degree of accuracy in the background estimates and potentially allow for additional systematic variations.
In the expansion of our systematics consideration a next logical step is the set of uncertainties related cosmic rays and their production of neutrinos.
Thankfully, MCEq allows arbitrary cosmic ray model fluxes and compositions as input, and has a range of hadronic models to select from. The effect of these systematics on the atmospheric neutrino flux can then be directly evaluated.
At the same time, the improved passing fractions calculation leverages this same flexibility so that these systematics may be taken into account for accompanying muons as well.

The MESE sample is not the only option for obtaining more information about cascades across a broader energy range.
The ``cascade'' sample provides similar information as MESE, but with a larger effective area and higher signal to background ratio.
However, it relies on non-veto methods to perform the selection.
Absent the ability to compartmentalize the detector response to muons and neutrinos, the estimation of backgrounds must rely on other methods than the calculation presented in \refsec{sec:passingfractions} or a more complicated extension of this.
Other available methods of modeling the interplay between related atmospheric muons and neutrinos and their combined effect on the event selection rely on air-shower simulations.
Air-shower simulations to this end have several downsides: they tend to be computationally expensive by comparison to signal simulation, must often be specialized for a particular event selection, are only available for a small subset of hadronic interaction models, and must be rerun for any hadronic interaction model changes.
For this reason it may prove difficult to add additional atmospheric systematic uncertainties to an analysis of the cascade sample.
An easier first step may be to simply use the less sensitive MESE sample, which has the option to account for these systematics with faster calculations.

The cascade dominated selections are largely comprised of $NC$, $\nu_e$ CC, and a sub-population of $\nu_\tau$ CC events.
Beyond this we hope to incorporate similar information from $\nu_\mu$ CC interactions.
This is available through the different track dominated event selections.
Two samples, the through-going muon (TGM) neutrino selection and the selection for matter enhanced oscillations with steriles (MEOWS), select for track-like events largely originating from muon neutrino charged current interactions in the Northern hemisphere.
Either of these samples would add much needed information about the astrophysical muon neutrino spectrum, allowing us to search for differences between the spectra of different flavors and to more precisely measure the flavor content of the flux.
Even under the assumption of an isotropic 1:1:1 astrophysical flux, there are benefits to including one of these samples in the analysis.
These samples extend between $\SI{100}\GeV$ and $\SI{10}\PeV$, and below a reconstructed energy of $\sim\SI{20}\TeV$ the flux is dominated by atmospheric neutrinos.
This large population of atmospheric neutrinos can help to pin down the unknown values of the atmospheric systematic parameters, thereby reducing uncertainty in the astrophysical measurements even for other samples.
Many events are shared between these two samples, and the differences between them largely center around a choice of selection techniques, differing reconstruction quality cuts, and slightly different zenith range.
The TGM sample potentially offers more sensitivity as it has a larger zenith range and higher effective area.
However, the MEOWS sample may be a good first step as its more stringent quality cuts may help with data/MC agreement and contamination from muon backgrounds near the horizon.

The two muon neutrino samples described above cover only the Northern hemisphere, so we would like to include a track sample with sensitivity in the Southern sky.
The enhanced starting track event selection (ESTES) works by taking the reconstructed event directions, computing the probability that a muon could sneak through the detector region behind the event vertex, and rejecting events where this probability is high.
This is similar to veto techniques as a sufficiently detailed veto definition that depends on the properties of the event could produce the same event selection.
With ESTES the main challenge for background estimation is accounting for the interplay between muon and neutrino event properties and how they may conspire to have the event accepted or rejected.
A sufficiently detailed $P_{\rm light}$ that depends on both muon and neutrino properties can account for this and allow us to use the calculation in \refsec{sec:passingfractions} for background estimation.
Although the effective area of ESTES is smaller than other selections when all directions are considers, it is uniquely able to obtain a relatively pure sample of astrophysical neutrinos in the Southern hemisphere that extends down to a few $\si\TeV$.

Selections mentioned above cover the categories of cascades and tracks across a wider range of neutrino energies and encompass the entire sky.
Finally the task remains to include information about tau neutrinos.
There are not any known properties of $\nu_\tau$ CC events that allow them to be readily separated from muon backgrounds and other types of neutrino interactions.
However, identifiers have been constructed to differentiate $\nu_\tau$ CC events from others assuming the events do not originate from the muon background.
Thus the most effective approach is to apply these identifiers to existing selections and statistically infer the properties of the tau neutrino flux rather than trying to separate the events outright.
Such identifiers use different methodologies but all hinge on the properties of a particular class of $\nu_\tau$ CC events where there are two physically distinct particle showers in the event, one from the initial neutrino DIS and the other from the tau decay.

By combining these event selections and identifiers an analysis can be constructed that is able to test for structure and discrepancies in the different channels while also obtaining more precise measurements of the astrophysical flux that are less limited by systematic uncertainties.

\subsection{Analysis improvements}
Following \refig{fig:analysis_overview} we can examine what components of the analysis may be improved.
In the previous subsection we have already added additional samples, improving the ``event selection''.
Event reconstruction can be improved, but this topic had not been the focus of this work and so we will ignore it for the moment.
The other interesting aspects that can be improved lie in the MC generation, likelihood function, and reweighting; in fact the three are related to one another.
Signal estimation is relatively optimal at this stage of IceCube's operation with the exception of some unmodeled detector and ice systematics.
Background estimation on the other hand remains challenging and is susceptible to a wider array of systematic effects.
This can be broken down into two categories, the direct contribution of muons to the event selection and the effect of related muons on the acceptance of atmospheric neutrinos.

For muons that directly contribute to observations the main problem is the huge number of muons in proportion to neutrinos.
Based purely on interaction cross section we expect the detection of approximately $10^6$ more muons than neutrinos.
Assuming the simulation of a single muon or neutrino event incurs similar computational cost, the simulation of background muons is vastly more expensive than simulation of neutrinos for equivalent periods of time.
The event selections must also be extremely efficient at rejecting muons to measure the neutrino flux, so for all the expensive background estimation we can muster there is often very little information regarding the muon background distribution in the sample.
This is exemplified in \reffig{fig:muons} which shows the expected distribution of muons in the HESE selection.
Most of the bins in this figure are empty, indicating inadequate simulation, and this a projection that reduces the number of bins by approximately a factor of three.
Clearly this situation can be improved.

The two main ways improve upon this without orders of magnitude more computing power are to improve simulation efficiency, and to make better use of available information.


