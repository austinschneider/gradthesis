\chapter{Searching for astrophysical neutrinos: High Energy Starting Events}

\section{Event selection\label{sec:selection}}
\begingroup
\graphicspath{{results/HESE_Final_Paper/}}
\chapter{Searching for astrophysical neutrinos: High Energy Starting Events}

\section{Event selection\label{sec:selection}}
\begingroup
\graphicspath{{results/HESE_Final_Paper/}}
\chapter{Searching for astrophysical neutrinos: High Energy Starting Events}

\section{Event selection\label{sec:selection}}
\begingroup
\graphicspath{{results/HESE_Final_Paper/}}
\chapter{Searching for astrophysical neutrinos: High Energy Starting Events}

\section{Event selection\label{sec:selection}}
\begingroup
\graphicspath{{results/HESE_Final_Paper/}}
\input{results/HESE_Final_Paper/sections/detectorselection/selection}
\endgroup

\section{Determination of atmospheric neutrino and muon backgrounds\label{sec:backgrounds}}
\begingroup
\graphicspath{{results/HESE_Final_Paper/}}
\input{results/HESE_Final_Paper/sections/backgrounds}
\endgroup

\section{Accounting for veto effects}
\endgroup

\section{Determination of atmospheric neutrino and muon backgrounds\label{sec:backgrounds}}
\begingroup
\graphicspath{{results/HESE_Final_Paper/}}
\chapter{Determination of atmospheric neutrino and muon backgrounds}

\section{Cosmic rays}

\section{Neutrino production in air-showers}

\section{Accounting for veto effects}

\section{Muon backgrounds}

\endgroup

\section{Accounting for veto effects}
\endgroup

\section{Determination of atmospheric neutrino and muon backgrounds\label{sec:backgrounds}}
\begingroup
\graphicspath{{results/HESE_Final_Paper/}}
\chapter{Determination of atmospheric neutrino and muon backgrounds}

\section{Cosmic rays}

\section{Neutrino production in air-showers}

\section{Accounting for veto effects}

\section{Muon backgrounds}

\endgroup

\section{Accounting for veto effects}
\endgroup

\section{Estimation of backgrounds}

In the search for astrophysical neutrinos, the predominant backgrounds are from atmospheric neutrinos and atmospheric muons.
Single atmospheric neutrinos are able to produce the same event signatures as astrophysical neutrinos because they are not fundamentally different astrophysical neutrinos.
Atmospheric neutrinos can only be distinguished from their astrophysical counterparts by examining the population properties, or by the fact that muons produced in the same air shower can also reach the detector simultaneously.
Muons from cosmic ray air showers differ from neutrinos in their event signature as they always produce light while entering the detector, while neutrinos can interact within the detector volume before any light is produced near the edge of the detector.
Muons can also only be observed up to a certain amount of overburden, beyond which they will lose their energy and decay before reaching the detector.
This means the up-going observation region is free from atmospheric muons.

\subsection{Neutrino background estimation}

Atmospheric neutrinos are predominantly produced by the decay of pions and kaons, which we shall call the ``conventional'' component.
At energies above $\SI{1}\TeV$ the spectrum of the conventional component is softer than the incident cosmic-ray spectrum by one unit in the spectral index, due to the interactions of these mesons in the atmosphere.
The conventional neutrino flux is largest at the horizon, $\cos\theta_z=0$~\cite{Gaisser:2002jj,Barr:2004br,Honda:2006qj,Petrova:2012qf} because the larger path length in the thin atmosphere increases the proportion of pions that can decay before interacting.
To model the conventional neutrino flux, we use a parameterization of the Honda {\it{}et al.} 2006 flux calculation~\cite{Honda:2006qj} given in~\cite{Montaruli:2011as} which at the highest energies uses the analytic parameterization of the neutrino flux in~\cite{Gaisser:2002jj}.
This calculation is tuned to match observations of atmospheric muons, which remain difficult to predict from first principles.

A sub-leading -- yet unobserved -- contribution due to charmed hadron decays is expected to be important above $\sim\SI{100}\TeV$~\cite{Bhattacharya:2015jpa}.
Since the charmed hadrons decay promptly and do not interact in the atmosphere at the energies relevant for this analysis, we call this the ``prompt'' component.
Thus, at these energies, the prompt component has a spectral index close to the incident cosmic-ray spectrum.
The prompt flux is also constant with respect to the cosine of the zenith angle, as the short decay time of the charmed hadrons removes the effect of different path lengths through the atmosphere.
To model the prompt neutrino flux at Earth's surface, we use the flux computed in~\cite{Bhattacharya:2015jpa}.

The angular and energy distribution of the initial atmospheric neutrino flux is modified with respect to the flux at Earth's surface because of absorption of high-energy neutrinos in the Earth.
We account for this using a dedicated Monte Carlo, similar to the one described in~\cite{Gazizov:2004va}, that simulates the propagation of neutrinos in the Earth.
In this Monte Carlo we use the isoscalar neutrino cross sections given in~\cite{CooperSarkar:2011pa} for the neutrino-nucleon interactions, and the Earth density model described in~\cite{Dziewonski:1981xy}.
Neutrino-electron scattering can be safely neglected except for resonant-W production~\cite{Glashow:1960zz} which we include.
We ignore the uncertainties on the Earth opacity as they are known to be sub-leading in this energy range~\cite{Gandhi:1995tf,CooperSarkar:2011pa,Vincent:2017svp}.

Modeling the atmospheric neutrino flux with these two components does not account for the contribution of $K_s$~\cite{Gaisser:2014pda}, which is $\sim \SI{10}\percent$ at $\SI{100}\TeV$ and well-within our uncertainties.
In order to account for uncertainties in the cosmic-ray flux~\cite{Dembinski:2017zsh} and hadronic interactions~\cite{Fedynitch:2012fs} we parameterize the atmospheric neutrino flux as
\noindent
\begin{linenomath*}
	\begin{equation}
	\begin{split}
	\phi_\nu^{\rm{atm}} =& \Phi_{conv} \bigg(\phi^\pi_\nu + R_{K/\pi} \phi^K_\nu\bigg) {\bigg(\frac{E_\nu}{E_0^c} \bigg)}^{-\Delta \gamma_{CR}} \\ &+ \Phi_{prompt} \phi^p_\nu {\bigg(\frac{E_\nu}{E_0^p} \bigg)}^{-\Delta \gamma_{CR}},
	\end{split}
	\label{eq:atm_flux_equation}
	\end{equation}
\end{linenomath*}
where $\phi^\pi_\nu$, $\phi^K_\nu$, and $\phi^p_\nu$ are the conventional pion, kaon, and prompt atmospheric neutrino fluxes at a neutrino energy $E_\nu$ respectively.

The parameters $\convnorm$ and $\promptnorm$ are normalizations for the conventional and prompt normalizations respectively; $\pik$ allows us to modify the relative kaon to pion contributions; and the $\crdeltagamma$ parameter allows for hardening or softening of the atmospheric neutrino component to account for uncertainties in the cosmic-ray flux slope.
These parameters are incorporated into the analysis as nuisance parameters with priors as summarized in \reftab{tbl:priors}.
Priors are selected either to be Gaussian or uniform if otherwise unspecified.
This analysis refrains from directly using prior information from other IceCube neutrino studies in order to provide results independent from other samples.
The conventional normalization Gaussian prior width is motivated by studies of the total uncertainty due to cosmic-ray and high-energy hadronic processes~\cite{Fedynitch:2012fs}.
The width of the cosmic-ray slope parameter has been chosen to be $0.05$ in order to accommodate values measured at intermediate~\cite{Karelin:2011zz} and high~\cite{Bartoli:2015fhw,Yoon:2017qjx,Alfaro:2017cwx} energies.
The corrections to the ratios of neutrinos-to-anti-neutrinos and kaon-pion yields uncertainty was estimated by comparing the expectation of different atmospheric neutrino calculations and picking a width that encompasses their prediction.
Details regarding these corrections and the different atmospheric neutrino flux calculations can be found in~\cite{CollinFluxes,Jones:2015bya}.
Finally, the parameters $E_0^c=\SI{2020}\GeV$ and $E_0^p=\SI{7887}\GeV$ are points of fixed differential flux for the conventional and prompt components.

This simple parameterzation of the atmospheric neutrino flux and its uncertainties is chosen because it models the primary systematic effects of physical uncertainties that are observable with this sample.
Of course this neglects other physical uncertainties, such as those regarding the hadronic interactions in cosmic ray air showers and the composition of the cosmic ray particles.
But, these effects do not produce modifications to the observations that can be statistically discerned with the amount of data available.
Later it will be shown that even the modeled systematic uncertainties have a small effect on astrophysical measurements compared to the statistical uncertainties.

% Table of all systematics and their priors if applicable
\begin{table*}[thb]
	\centering
	% systematic parameter & prior type & mean & sigma & min & max
	\begin{tabular}{l rrr}
		%\toprule
		%& \multicolumn{5}{c}{Prior information} \\
		%\cmidrule{2-6}
		Parameter & Prior (constraint) & Range & Description \\
		\toprule
		\multicolumn{1}{l }{\textbf{Astrophysical neutrino flux:}} & & & \\
		$\astronorm$ & - & $[0,\infty)$ & Normalization scale\\
		$\astrodeltagamma$ & - &  $(-\infty,\infty)$ & Spectral index\\
		%$\theta_a$ & Uniform &  &  & 0 & 1 \\ \hline
		%$\theta_b$ & Uniform &  &  & -1 & 1 \\ \hline
		%${2\nu/\left(\nu+\bar{\nu}\right)}_\texttt{astro}$ & - & $[0,2]$\\
		&&\\
		\midrule
		\multicolumn{1}{l }{\textbf{Atmospheric neutrino flux:}} & & &\\
		$\convnorm$ & $1.0\pm0.4$ & $[0, \infty)$ & Conventional normalization scale\\
		$\promptnorm$ & - & $[0, \infty)$ & Prompt normalization scale\\
		$\pik$ & $1.0\pm0.1$ & $[0, \infty)$ & Kaon-Pion ratio correction\\
		$\atmonunubar$ & $1.0\pm0.1$ & $[0,2]$ & Neutrino-anti-neutrino ratio correction\\
		&&\\
		\midrule
		\multicolumn{1}{l }{\textbf{Cosmic ray flux:}} & & &\\
		$\crdeltagamma$ & $0.0\pm 0.05$ & $(-\infty,\infty)$ & Cosmic-ray spectral index modification\\
		$\muonnorm$ & $1.0\pm 0.5$ & $[0,\infty)$ & Muon normalization scale\\
		&&\\
		\midrule
		\multicolumn{1}{l }{\textbf{Detector:}} & & &\\
		$\domeff$ & $0.99 \pm 0.1$ & $[0.80, 1.25]$ & Absolute energy scale\\
		$\holeice$ & $0.0 \pm 0.5$ & $[-3.82, 2.18]$ & DOM angular response\\
		$\anisotropy$ & $1.0 \pm 0.2$ & $[0.0, 2.0]$ & Ice anisotropy scale\\
	\end{tabular}
	\internallinenumbers
	\caption{\textbf{\textit{Analysis model parameters for the single power-law astrophysical model.}} Prior probabilities (constraints) for analysis parameters used in Bayesian (frequentist) analyses respectively.
		Priors (constraints) on the parameter are either uniform or Gaussian.
		Where applicable, the mean, standard deviation, and bounds are given.}\label{tbl:priors}
\end{table*}


\subsection{Atmospheric neutrino passing fractions}

As noted in~\cite{Schonert:2008is}, muons produced in the same air-shower may trigger the detector veto in coincidence with the neutrino interaction.
Dedicated simulations of cosmic ray air showers where neutrino interactions are forced can provide observational predictions for the distribution of these combines muon+neutrino events.
However, available simulation techniques to create these estimates remain prohibitively expensive for the desired accuracy.
Instead, we look to model this effect by computing an average efficiency with respect to the case where no muons reach the detector.
To account for this when weighting the neutrino-only simulation each atmospheric neutrino flux component, $i$, is multiplied by an efficiency.
This efficiency is referred to as the ``atmospheric neutrino passing fraction'', and denoted by $\mathcal{P}^{i,\alpha}_{passing}$, for each neutrino flavor $\alpha$.

The passing fraction depends on the details of air shower development, which includes modeling of the hadronic interaction, and the energy losses of muons in the material between the shower and the detector. 
Air-shower properties are averaged over in this calculation, but the passing fraction is still a probability that is conditional on the neutrino properties.
Of particular interest are the potential systematic dependencies of the passing fraction, which include the cosmic ray spectrum and composition, the hadronic interaction model, muon energy losses, and the atmospheric density model.
The calculation of the passing fraction is designed so that these systematic dependencies may be changed between different alternatives.
Additionally, the detector response enters in only one place in the formalism as a probability of muon rejection.
This allows for simple variation of the detector response modeling in a generic way.
To perform this calculation a framework was developed that leverages the Matrix Cascade Equation (MCEq) package in key portions of the calculation.
Validation of the computed passing fractions is performed against detailed air shower simulations with COsmic Ray SImulations for KAscade (CORSIKA), and excellent agreement is found as a result.

% TODO move this text to the section where we discuss analysis specifics
%In this approach the neutrino properties are known from simulation, and it is sought to average over all the potential properties of the cosmic ray air showers from which the neutrino could have been produced in.
%Ideally this average over air shower properties should be computed for each neutrino position, direction, and energy because the detector response can vary with all six of these parameters.
%However, not all of these properties are used directly in the analysis nor does the detector response depend equally strongly on all of these parameters.
%For this reason when performing the calculation of the efficiency, only the neutrino energy, zenith angle, and depth upon intersection with the detector are considered.
%Other properties of the neutrino are averaged over.
%Additionally, in the characterization of the detector response to muons, only dependence on the muon energy and depth are considered.
%Thus, the computed passing fraction depends on the neutrino energy, the zenith angle, and the incident depth in the detector.
%However, the detector response and neutrino properties can be factorized so that the problem can be discussed more generally.
%In previous analyses, the passing fractions were calculated using an extension of the method described in~\cite{Schonert:2008is} and bounded at $\SI{10}\percent$; details of the method are provided in~\cite{Aartsen:2013jdh}.
%Cosmic ray simulations remain a computationally prohibitive way of accounting for the effects of accompanying muons, so we still rely on calculations of the average passing rate.
%In this analysis, we use a new calculation given in~\cite{Arguelles:2018awr} that allows for different cosmic-ray and hadronic models to be used; more importantly for this analysis any parameterization of the detector veto response to muons can be used in the calculation, as opposed to just an energy threshold.
%This capability allows us to more accurately model the detector response to atmospheric neutrinos.
%In \reffig{fig:P_light} we show the probability that a muon will pass the veto as a function of the true muon incident energy for different detector depths.

Departure from the lone-neutrino case is based on the coincidence of a muon and neutrino from the same air-shower that occurs in both time and direction.
Thus, $\Prob_{\rm reach}\left(\Emf \, | \, \Emi , \, X_\mu, \right)$ is naturally a key component of the calculation, which describes the probability of a muon with initial energy $\Emi$ and slant depth $X_\mu$ to reach the detector with energy $\Emf$.
As the IceCube detector is deep underground, the contribution to slant depth from the atmosphere is very small compared to that of the Earth, and so can be neglected.
With only the slant depth of the Earth contributing, $X_\mu$ is fully specified by the zenith angle $\theta_z$ in detector coordinates and the depth $d$.
Stochasticity of the muon energy losses is accounted for by the distribution $\Prob_{\rm reach}$, which is computed by tabulating simulations of muons propagated in ice by the software $\MMC$.
Using the conditional probability distribution $\Prob_{\rm reach}$ instead of average muon behavior ensures that contributions from the distribution tails are accounted for.
These contributions from the tails become important for very large slant depths, where muons on average do not reach the detector.
Departure from the neutrino-only case requires both the muon to reach the detector, but also for the muon to trigger a response, in this case for the muon to {\it light} the veto.
The quantity $\Prob_{\rm light}(\Emf,d)$ is the probability that a muon of energy $\Emf$ and depth $d$ at the detector boundary {\it lights} the veto, such that the event is rejected.
In practice, this $\Prob_{\rm light}$ is computed by tabulating detector simulations of atmospheric muons and incorporating the specifics of the event selection.
With both $\Prob_{\rm light}$ and $\Prob_{\rm reach}$ in hand, the probability of detecting an incident muon $\Prob_{\rm det}$ can be computed as 

\begin{equation}
\label{eq:PrPl}
\Prob_{\rm det}\left(\Emi , \, X_\mu(\theta_z, d), d\right) \equiv \int d\Emf \, \Prob_{\rm reach}\left(\Emf \, | \, \Emi , \, X_\mu(\theta_z, d)\right) \, \Prob_{\rm light} (\Emf, d) ~.
\end{equation}

Note that losses also depend on the medium and not only on the slant depth, whose dependence on $\theta_z$ we write explicitly. If the medium surrounding the detector is homogeneous, the dependence on $X_\mu$ can be exchanged for distance without loss of generality.
From here on to simplify the notation, the dependence on $X_\mu$ will be replaced by a dependence on $\theta_z$ and the dependence on $d$ will be neglected.
These dependencies can be added back in at the end of the calculation without any change to the result.

% TODO
\textbf{TODO: Add a plot of $\Prob_{\rm reach}$, $\Prob_{\rm light}$, and $\Prob_{\rm det}$}

The atmospheric neutrino passing fraction is defined as the probability for an atmospheric neutrino to not be accompanied by a muon which is detected from the same air shower. This probability is denoted $\Prob_{\rm pass}$~\cite{Schonert:2008is, Gaisser:2014bja} and can be written as the ratio
\begin{equation}
\Prob_{\rm pass} (E_\nu, \theta_z) = \frac{\phi_\nu^{\rm pass}(E_\nu, \theta_z)}{\phi_\nu(E_\nu, \theta_z)} ~,
\end{equation}
where $E_\nu$ is the neutrino energy, $\theta_z$ is the zenith angle, $\phi_\nu^{\rm pass}$ is the differential flux of atmospheric neutrinos accompanied by muons that are detected, and $\phi_\nu$ is the total differential atmospheric neutrino flux.

\subsubsection{$\nu_e$ passing fraction}
Electron neutrinos (or antineutrinos) are produced alongside a positron (or electron) in the decay of their parent particle, for example $K^+ \rightarrow \pi^0 + e^+ + \nu_e$.
Rarely does an electron neutrino have a sibling muon, as this would involve a lepton flavor violating process.
Instead, muons that accompany electron neutrinos are produced in other branches of the air shower.
Because the different branches of the shower are uncorrelated, to first order the average properties of muons in a prototypical air shower can be used for the passing fraction calculation.
Consider a shower from a single cosmic-ray primary particle of type $A$ with energy $\ECR$.
The average neutrino yield $\frac{dN_{A, \nu}}{dE_\nu}(\ECR, E_\nu, \theta_z)$ from such a shower can be computed with $\MCEq$.
So the atmospheric neutrino flux can be written as 
\begin{equation}
\label{eq:nuphi}
\phi_\nu(E_\nu, \theta_z)  = \sum_A \int d\ECR \, \frac{dN_{A, \nu}}{dE_\nu}(\ECR, E_\nu, \theta_z) \, \phi_A(\ECR)~,
\end{equation}
where $\phi_A(\ECR)$ is the flux of primary cosmic rays of type A.
the passing fraction can be obtained by modifying the integrand of Eq.~\ref{eq:nuphi} so that it is weighted by the Poisson probability $Pzmproto$ of detecting an accompanying muon.
In this way the uncorrelated passing fraction can be written as
\begin{equation}
\label{eq:PuncorGJKvS}
\Prob_{\rm pass}^{\rm uncor, GJKvS}(E_\nu, \theta_z)  =  \frac{1}{\phi_\nu(E_\nu, \theta_z)} \sum_A \int d\ECR \, \frac{dN_{A, \nu}}{dE_\nu}(\ECR, E_\nu, \theta_z) \, \phi_A(\ECR) \, \Pzmproto \left(N_\mu = 0  ; \bar N_{A, \mu}^{\rm GJKvS}(\ECR, \theta_z)\right)~,
\end{equation}
where $\Pzmproto \left(N_\mu = 0  ; \bar N_{A, \mu}^{\rm GJKvS}(\ECR, \theta_z)\right) = \exp{-\bar N_{A, \mu}^{\rm GJKvS}(\ECR, \theta_z)}$,
and  $\bar N_{A, \mu}^{\rm GJKvS}(\ECR, \theta_z)$ is the average number of muons that are detected from an air shower.
This average number of muons is computed using the yield of muons from the air shower $dN_{A, \mu}/d\Emi(\ECR, \Emi)$ and the probability of detecting those muons, such that
\begin{equation}
\label{eq:Nmu}
\bar N_{A, \mu}^{\rm GJKvS}(\ECR,\theta_z) = \int d\Emi \, \frac{dN_{A, \mu}}{d\Emi}(\ECR, \Emi,  \theta_z) \, \Prob_{\rm det}\left(\Emi , \theta_z\right) ~,
\end{equation}
Some terms in the above equations are labeled with ${\rm GJKvS}$ because these quantities are calculated in the same manner as was done in~\cite{Gaisser:2014bja} under certain choices for $\Prob_{\rm det}$.
Namely, when $\Prob_{\rm light}$ is defined as a Heaviside function with a boundary at $\Emf = \SI{1}\TeV$, and $\Prob_{\rm reach}$ is defined as a delta function that matches the initial muon energy to the median final muon energy.

\begin{figure}
	\centering
	\subfloat{
		\includegraphics[width=0.45\linewidth]{results/passing_fractions_paper/fig/fig4_nue_conv_murange_vs_avg}
	}
	\subfloat{
		\includegraphics[width=0.45\linewidth]{results/passing_fractions_paper/fig/fig4_nue_prompt_murange_vs_avg}
	}    
	\caption{\textbf{\textit{Passing fractions: effect of the treatment of muon losses in ice.}} Results are shown for three values of $\cos\theta_z$ (from top to bottom): 0.25 (blue), 0.45 (green), and 0.85 (orange); using the full muon range distribution (solid) or the median muon range (dashed). Results from the \CORSIKA{} simulation are shown as crosses, with statistical error bars only. In all cases, the H3a primary cosmic-ray spectrum~\cite{Gaisser:2011cc}, the SIBYLL~2.3 hadronic-interaction model~\cite{Engel:2015dxa, Riehn:2015oba} and the MSIS-90-E atmosphere-density model at the South Pole on July 1, 1997~\cite{Labitzke:1985, Hedin:1991} are used. A depth in ice of $d_{\rm det} = \SI{1.95}\km$ (like the center of IceCube) and a Heaviside $\Prob_{\rm light}(\Emf) = \Theta(\Emf - 1\,{\rm TeV})$ are assumed. \textit{Left panel:} Conventional $\nu_e$ passing fraction. \textit{Right panel:} Prompt $\nu_e$ passing fraction.
	}
	\label{fig:nue_passing-preach-effect}
\end{figure}

In Eq.~\ref{eq:Nmu} the energy available to other branches of the shower to produce uncorrelated muons is over estimated as some energy must be reserved for producing the electron neutrino or rather the branch of the shower that produces the electron neutrino.
A complete modeling of the connection between the three relevant particles (cosmic ray primary, the electron neutrino, and muons) is cumbersome, as it would involve modeling the shower branch producing the electron neutrino in its entirety and then separately modeling the uncorrelated branch.
However, a simple approximation can be made that at least accounts for the energy necessary to produce the electron neutrino.
The electron neutrino must be produced by a parent particle $p$ that has a kinematically allowed energy $E_p$, so the remaining energy is $\ECR-E_p$.
The muon yield is then modeled using the average behavior of a shower that begins with energy $\ECR-E_p$.
Because we are now concerned with the parent particle of the electron neutrino, the slant depth through which this parent particle travels must be considered as this will affect the probability that it decays to a neutrino instead of interacting.


\subsubsection{$\nu_\mu$ passing fraction}
\subsubsection{Calculation improvements}
\subsubsection{Calculation systematics}

% Plot of p_light
\begin{figure}
	\centering
	\includegraphics[width=\linewidth]{results/HESE_Final_Paper/figures/plight}
	\internallinenumbers
	\caption{\textbf{\textit{Muon veto passing fraction.}} Each line shows the fraction of muons of a given energy at the detector edge, $E_\mu$, that pass without triggering the veto when entering the detector at a particular depth.
		Three depths are shown: 1500, 1950, and 2300 meters from the surface; with lines of darkening color as the depth increases.
		The veto efficiency increases with the muon energy.
		Differences at various depths are due to the changing ice properties, and varying acceptance as a function of depth due to the asymmetric structure of the veto region.
		At all depths a sigmoid function is fit to the results of muon simulation.
		Above $\sim\SI{100}\TeV$ the passing fraction is extrapolated.}\label{fig:P_light}
\end{figure}

Using the passing fractions in \reffig{fig:P_light} as input and the \nuveto{} code provided in~\cite{Arguelles:2018awr} we calculate the atmospheric passing fraction for each component and flavor using the Hillas-Gaisser H3a~\cite{Gaisser:2013bla,Gaisser:2011cc,Hillas:2006ms} model for the incident cosmic-ray spectra and SIBYLL2.3c~\cite{Riehn:2017mfm} for the hadronic interactions in the air shower.
Using passing fractions derived from alternative cosmic-ray and hadronic interaction models has sub-leading effects in the determination of the astrophysical flux~\cite{Arguelles:2018awr}.
In this work, this was studied by repeating the analysis for different passing fractions that arise from a given combination of cosmic-ray spectrum and hadronic model for a variety of spectra and models that are available in the literature.
We found that the inclusion of these effects in addition to other discrete ice choices mentioned later in \refsec{sec:detector_systematics} increases the reported uncertainty of the astrophysical parameters by at most $\SI{20}\percent$ with respect to errors computed without these effects.
For this reason, these effects are not included in the analysis and are not reflected in the reported errors of any model parameters.
In \reffig{fig:passingfraction} we show the passing fractions for the conventional and prompt neutrino components.
In these figures the left, center, and right panels correspond to $\cos\theta_z$ values of 0.1, 0.3, and 0.9 respectively; the solid lines correspond to muon neutrinos and the dashed lines to electron neutrinos.
From the progression of the panels from left to right, one can see the passing fractions become smaller as one approaches vertical directions.
Vertical muons have the highest probability of reaching the detector, as the overburden they pass through is the smallest.
Though not shown in this figure, the conventional passing fractions differ from neutrinos to anti-neutrinos, see~\cite{Arguelles:2018awr} for details; the appropriate passing fractions are used in this analysis.
\reffigs{fig:conventional_distribution}{fig:prompt_distribution} show the distributions of conventional and prompt neutrinos respectively after this correction is applied.
This reduction in atmospheric background accounts for much of the sensitivity of this analysis to the astrophysical neutrino flux, as the observed down-going atmospheric fluxes in IceCube would otherwise be comparable in magnitude and remain similar in their angular distribution.
This is best seen when comparing the atmospheric fluxes before and after the veto to the measured astrophysical flux as shown in \reffig{fig:neutrino_spectrum}.

% Plot of the passing fraction for different heights (one line per height) (one panel for each costh [3 values])
\begin{figure*}
	\centering
	\subfloat{\includegraphics[width=0.3\linewidth]{results/HESE_Final_Paper/figures/conv_0_1_passing_fraction}}
	\subfloat{\includegraphics[width=0.3\linewidth]{results/HESE_Final_Paper/figures/conv_0_3_passing_fraction}}
	\subfloat{\includegraphics[width=0.3\linewidth]{results/HESE_Final_Paper/figures/conv_0_9_passing_fraction}} \\
	\subfloat{\includegraphics[width=0.3\linewidth]{results/HESE_Final_Paper/figures/prompt_0_1_passing_fraction}}
	\subfloat{\includegraphics[width=0.3\linewidth]{results/HESE_Final_Paper/figures/prompt_0_3_passing_fraction}}
	\subfloat{\includegraphics[width=0.3\linewidth]{results/HESE_Final_Paper/figures/prompt_0_9_passing_fraction}}
	\internallinenumbers
	\caption{\textbf{\textit{Conventional and prompt atmospheric component passing fraction.}}
		The top row of plots shows the atmospheric neutrino passing fraction as a function of the neutrino energy for a flux of neutrinos originating from pions and kaons, assuming the Hillas-Gaisser H3a~\cite{Gaisser:2013bla,Gaisser:2011cc,Hillas:2006ms} cosmic-ray model and SIBYLL 2.3c~\cite{Riehn:2017mfm} hadronic interaction model.
		While the bottom row of plots shows the atmospheric neutrino passing fraction for a flux of neutrinos originating from charmed hadrons under the same assumptions.
		Solid lines correspond to muon neutrinos and dashed lines to electron neutrinos.
		The different colors, from darkest to lightest, are for three different detector depths: 1350, 1950, and 2550 meters below the surface.
		The left, center, and right panel correspond to cosine of the zenith angles 0.1, 0.3, and 0.9 respectively (or zenith angles of $\SI{84.3}\degree$, $\SI{72.5}\degree$, and $\SI{25.8}\degree$).}\label{fig:passingfraction}
\end{figure*}

\begin{figure}
	\centering
	\includegraphics[width=\linewidth]{results/HESE_Final_Paper/figures/diffuse_hist_all_conv}
	\internallinenumbers
	\caption{\textbf{\textit{Expected distribution of atmospheric neutrinos produced by pions and kaons in the sample.}} Distribution of neutrinos that pass the veto as a function of the deposited energy and the cosine of the zenith angle assuming nominal values for the nuisance parameters.
		The dashed line at $\SI{60}\TeV$ marks the low energy cut of the analysis.
		Suppression in the down-going region is due to the veto, while suppression in the up-going region is due to absorption of neutrinos in the Earth.}\label{fig:conventional_distribution}
\end{figure}

\begin{figure}
	\centering
	\includegraphics[width=\linewidth]{results/HESE_Final_Paper/figures/diffuse_hist_all_prompt}
	\internallinenumbers
	\caption{\textbf{\textit{Expected distribution of atmospheric neutrinos produced by charmed hadrons in the sample.}} Distribution of neutrinos that pass the veto as a function of the deposited energy and the cosine of the zenith angle assuming nominal nuisance parameters and the BERSS flux calculation for neutrinos from  charmed hadrons~\cite{Bhattacharya:2015jpa}.
		The dashed line at $\SI{60}\TeV$ marks the low energy cut of the analysis.
		Suppression in the down-going region is due to the veto, while suppression in the up-going region is due to absorption of neutrinos in the Earth.}\label{fig:prompt_distribution}
\end{figure}

\begin{figure}
	\centering
	\includegraphics[width=\linewidth]{results/HESE_Final_Paper/figures/neutrino_spectrum}
	\internallinenumbers
	\caption{\textbf{\textit{All-sky astrophysical neutrino flux compared to down-going atmospheric neutrino fluxes before and after the veto.}}
		The atmospheric neutrino fluxes considered in this analysis are shown as dashed lines.
		The solid lines show the product of the atmospheric flux with the passing fraction averaged over depth at a zenith angle of $\SI{0}\degree$.
		The frequentist segmented power-law fit of the all-sky astrophysical flux assuming isotropy as described in \refsec{sec:generic_models} is shown in black.
		This comparison demonstrates the effect of the veto in the down-going region, where it is strongest.
		The suppression of the atmospheric flux becomes weaker towards the horizon, and is not present in the up-going region.
		The dashed lines labelled ``before-veto'' are equivalent to the up-going atmospheric fluxes, with or without the veto, neglecting Earth absorption effects.}
	\label{fig:neutrino_spectrum}
\end{figure}

\subsection{Muon background estimation}
Finally, there is also the possibility of single muons that trigger the event selection without a neutrino interaction in the detector and still pass the veto.
The shape of the atmospheric muon and neutrino fluxes are closely related to each other, and bounded by the cosmic-ray flux so that they must be steeply falling.
The interaction of muons in the atmosphere and ice further softens the muon spectrum from that of cosmic rays.
Although there is uncertainty in the shape of the muon spectrum, the yield of muons from cosmic-ray air showers has more significant modelling uncertainties that stem from uncertainties in the hadronic interaction cross sections~\cite{Pierog:2017nes} and the cosmic-ray composition~\cite{Bluemer:2009zf}.
As we lack the capability to parameterize both the uncertainty in shape and normalization from first principles, we turn to data-driven techniques to constrain the size of this background.
Unfortunately, the data-driven techniques available do not provide us with enough events to determine the shape of the muon background.
For this reason we take a pragmatic approach to treat the muon component.
We use a simulation estimate of the muon flux shape which provides a reasonable estimate for a steeply falling muon spectrum, but neglects shape uncertainties.
The normalization is then constrained using a procedure that tags background muons in data.
The spectrum of atmospheric muons from cosmic-ray air showers is modelled by a parameterization of muons from air showers simulated with the \CORSIKA~\cite{Heck:1998vt} package assuming the Hillas-Gaisser H4a~\cite{Gaisser:2013bla} cosmic-ray flux model and SIBYLL 2.1~\cite{Ahn:2009wx} hadronic model.
A dedicated single muon simulation, called \MUONGUN~\cite{jvsthesis}, is weighted to this flux. 
%Due to the uncertainties in the muon yield of cosmic-ray air showers we use a data-based prior to constrain its normalization and only use the shape from simulation.
To construct the data based prior, a second veto layer inside the original outer veto layer is introduced.
Events that trigger the outer veto layer, but do not trigger this second inner veto layer, are tagged as muons that pass the inner veto.
The muon normalization from simulation is re-scaled from $N_\MUONGUN$ to $2.1\cdot N^\mu_\textmd{tagged}$ to match the number of tagged muons while accounting for the relative size of the fiducial volumes.
Thus, the baseline expected muon flux is given by
\begin{linenomath*}
	\begin{equation}
	\begin{split}
	\frac{d^3\Phi}{d E_\mu d \theta_{z,\mu} d d_\mu} ={}& \frac{d^3\Phi_\texttt{GaisserH4a}}{d E_\mu d \theta_{z,\mu} d d_\mu}(E_\mu, \theta_{z,\mu},d_\mu)\\* & \cdot \frac{2.1 \cdot N^\mu_\textmd{tagged}}{N_\MUONGUN},
	\end{split}
	\label{eq:muon_scaling}
	\end{equation}
\end{linenomath*}
where $\Phi_\texttt{GaisserH4a}$ is the aforementioned parameterization; and $E_\mu$, $\theta_{z,\mu}$, and $d_\mu$ are the muon energy, zenith, and depth at injection respectively.
In \reftab{tbl:tag_muons} we list the number of tagged muons observed per year; in total 17 muons were observed.
The expected distribution of passing atmospheric muon events is shown in \reffig{fig:muons} as a function of the deposited energy and reconstructed cosine of the zenith angle.
The prior on the atmospheric muon rate is chosen to be Gaussian with a $\SI{50}\percent$ standard deviation, this encompasses the statistical uncertainty of our muon background measurement.

\begin{figure}
	\centering
	\includegraphics[width=\linewidth]{results/HESE_Final_Paper/figures/diffuse_hist_all_muons}
	\internallinenumbers
	\caption{\textbf{\textit{Expected distribution of atmospheric muons in the sample.}} Distribution of muons that pass the veto as calculated with \MUONGUN~as a function of the deposited energy and the cosine of the zenith angle.
		The normalization is set to match the data driven sub-detector study.
		The dashed line at $\SI{60}\TeV$ marks the low energy cut of the analysis.}\label{fig:muons}
\end{figure}

\begin{table}
	\centering
	% year & number of tagged muons
	\begin{tabular}{l r}
		\toprule
		Season & $N^\mu_{tagged}$ \\
		\midrule
		2010 & 2 \\
		2011 & 1 \\
		2012 & 1 \\
		2013 & 1 \\
		2014 & 2 \\
		2015 & 6 \\
		2016 & 2 \\
		2017 & 2 \\
		\midrule
		Total & 17 \\
		\bottomrule
	\end{tabular}
	\internallinenumbers
	\caption{\textbf{\textit{Number of tagged muons per season.}}
		Table shows the number of tagged muons used to construct the muon normalization prior.
		The first season, 2010, used a partial IceCube configuration with 79 strings, the rest of the seasons took data with the full configuration of 86 strings.
		The larger number of tagged muons in the 2015 season is believed to be a statistical fluctuation.
		The last season, 2017, represents only a partial year of data taking in this paper as the 2017 data processing was not yet completed at the time of this analysis.}\label{tbl:tag_muons}
\end{table}
