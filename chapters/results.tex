\chapter{Analysis}

The goal of the analysis in this chapter is to characterize the astrophysical neutrino spectrum using a high-purity sample of events.
On its own, this may not provide the great insight that we desire into larger problems of astropysical neutrino origin or cosmic ray acceleration mechanisms.
However, the detailed measurements performed here are a vital piece of the puzzle that may help us to answer these larger questions in the long run.
In pursuit of these measurements a handful of new techniques were also developed, and these will certainly improve the other measurements we make going forward.
This chapter describes the specific components that are prerequisite to a thorough analysis of the data, some of which have more general discussions in previous chapters.
After these components are outlined, the wide variety of tests and measurements of the astrophysical flux are explored and their results examined.

\section{Event selection}
To search for astrophysical neutrinos one first needs to sift through an comparatively immense background of events.
There are several event selections which are capable of doing this.
These different methods target different regions of the direction, energy, and flavor space.
The High Energy Starting Event (HESE) selection provides sensitivity above $\sim\SI{100}\TeV$ to astrophysical neutrinos of all flavors, across the entire sky.
Sensitivity to muon neutrinos is reduced with respect to electron and tau neutrinos.
This is primarily an artifact of muon neutrino CC interactions producing a muon, which lowers the amount of light observed for a given neutrino energy.
Further details of the event selection are provided in~\refsec{sec:selection}.

\section{Reconstruction and simulation}
\begingroup
\graphicspath{{results/HESE_Final_Paper/}}
\chapter{Event reconstruction and simulation}\label{chapter:reconstruction}
\begingroup
\graphicspath{{results/HESE_Final_Paper/}}
\chapter{Event reconstruction and simulation}\label{chapter:reconstruction}
\begingroup
\graphicspath{{results/HESE_Final_Paper/}}
\chapter{Event reconstruction and simulation}\label{chapter:reconstruction}
\begingroup
\graphicspath{{results/HESE_Final_Paper/}}
\input{results/HESE_Final_Paper/sections/reconstruction}
\endgroup
\endgroup
\endgroup
\endgroup

\section{Systematic uncertainties and statistical treatment\label{sec:uncertainties}}
\subsection{Detector systematic uncertainties\label{sec:detector_systematics}}
\begingroup
\graphicspath{{results/HESE_Final_Paper/}}
\input{results/HESE_Final_Paper/sections/uncertainties/systematics}
\endgroup

\subsection{Statistical treatment\label{sec:statistics}}
\begingroup
\graphicspath{{results/HESE_Final_Paper/}}
\chapter{Dealing with limited simulation samples}


\endgroup

\section{Characterization of the astrophysical neutrino flux\label{sec:diffuse}}
\begingroup
\graphicspath{{results/HESE_Final_Paper/}}
\input{results/HESE_Final_Paper/sections/diffuse/diffuse}
\endgroup

\subsection{Generic models\label{sec:generic_models}}
\begingroup
\graphicspath{{results/HESE_Final_Paper/}}
\input{results/HESE_Final_Paper/sections/diffuse/generic_models}
\endgroup

\subsubsection{Single power-law flux\label{sec:spl}}
\begingroup
\graphicspath{{results/HESE_Final_Paper/}}
\input{results/HESE_Final_Paper/sections/diffuse/spl}
\endgroup

\subsubsection{Double power-law flux\label{sec:dpl}}
\begingroup
\graphicspath{{results/HESE_Final_Paper/}}
\input{results/HESE_Final_Paper/sections/diffuse/dpl}
\endgroup

\subsubsection{Single power law with spectral cutoff\label{sec:cutoff}}
\begingroup
\graphicspath{{results/HESE_Final_Paper/}}
\input{results/HESE_Final_Paper/sections/diffuse/cutoff}
\endgroup

\subsubsection{Log-parabola flux\label{sec:log_parabola}}
\begingroup
\graphicspath{{results/HESE_Final_Paper/}}
\input{results/HESE_Final_Paper/sections/diffuse/lppl}
\endgroup

\subsubsection{Segmented power-law flux\label{sec:unfolding}}
\begingroup
\graphicspath{{results/HESE_Final_Paper/}}
\input{results/HESE_Final_Paper/sections/diffuse/segmented}
\endgroup

\subsection{Atmospheric flux from charmed hadrons\label{sec:prompt}}
\begingroup
\graphicspath{{results/HESE_Final_Paper/}}
\input{results/HESE_Final_Paper/sections/diffuse/prompt}
\endgroup

\subsection{Source-specific models\label{sec:specific_models}}
\begingroup
\graphicspath{{results/HESE_Final_Paper/}}
\input{results/HESE_Final_Paper/sections/diffuse/specific_models}
\endgroup