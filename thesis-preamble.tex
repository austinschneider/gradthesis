\usepackage{uwthesis}

%\usepackage[caption=false]{subfig}
%\usepackage{subfigure}
\usepackage{graphicx}
\usepackage{bm}
\usepackage{amssymb}
\usepackage{amsmath}
\usepackage{amsfonts}
\usepackage[utf8]{inputenc}
\usepackage{placeins}
% defines table rules for professional looking tables
\usepackage{booktabs}
\usepackage{gensymb}
\usepackage[colorlinks=true,allcolors=blue]{hyperref}
\usepackage{microtype}
\usepackage{multirow, makecell}
\usepackage[capitalise]{cleveref}
\usepackage[utf8]{inputenc}
\usepackage{listings}
\usepackage{siunitx}
%\sisetup{detect-weight=true, detect-family=true}
\sisetup{detect-all = true}
\usepackage{xspace}
\usepackage{comment}
\usepackage{ragged2e}
\usepackage{lineno}
\usepackage{subfiles}
\usepackage{textcomp}
\usepackage{rotating}
\usepackage{subcaption}
\usepackage[compat=1.1.0]{tikz-feynman}

\newcommand{\Python}{\texttt{Python}}
\newcommand{\MCEq}{\texttt{MCE{\scriptsize Q}}}
\newcommand{\MMC}{\texttt{MMC}}
\newcommand{\PROPOSAL}{\texttt{PROPOSAL}}
\newcommand{\CORSIKA}{\texttt{CORSIKA}}
\newcommand{\MUONGUN}{\texttt{MUONGUN}}
\newcommand{\PHOTOSPLINE}{\texttt{PHOTOSPLINE}}
\newcommand{\like}{\mathcal{L}}
\newcommand{\likeSAY}{\mathcal{L}_{\textrm{Eff}}}
\newcommand{\pdfSxi}{S(\vec{x}_i)}
\newcommand{\TS}{\mathrm{TS}}
\newcommand{\Fermi}{\textit{Fermi}}
\newcommand{\nuveto}{{\LARGE $\nu$}\texttt{eto}}

\DeclareMathOperator*{\argmax}{arg\,max}
\DeclareMathOperator*{\argmin}{arg\,min}

\newcommand{\refeq}[1]{Eq.~(\ref{#1})}
\newcommand{\refeqs}[2]{Eqs.~(\ref{#1})~and~(\ref{#2})}
\newcommand{\refeqss}[3]{Eqs.~(\ref{#1}), (\ref{#2})~and~(\ref{#3})}
\newcommand{\reffig}[1]{Fig.~\ref{#1}}
\newcommand{\reffigs}[2]{Figs.~\ref{#1}~and~\ref{#2}}
\newcommand{\refsec}[1]{Section~\ref{#1}}
\newcommand{\refsecs}[5]{Sections~\ref{#1},~\ref{#2},~\ref{#3},~\ref{#4},~and~\ref{#5}}
\newcommand{\refappsec}[1]{Appendix Section~\ref{#1}}
\newcommand{\refapp}[1]{Appendix~\ref{#1}}
\newcommand{\reftab}[1]{Table~\ref{#1}}
\newcommand{\refref}[1]{Ref.~\cite{#1}}
\newcommand{\refrefs}[2]{Refs.~\cite{#1}~and~\cite{#2}}

\newcommand{\vectheta}{\vec{\theta}}
\newcommand{\vecw}{\vec{w}}
\newcommand{\prob}{\mathcal{P}}
\newcommand{\gprob}{\mathcal{G}}
\newcommand{\meanl}{\mathcal{L}_{\textmd{Mean}}}
\newcommand{\mcl}{\like_\textmd{Eff}}
\newcommand{\gl}{\like_\textmd{G}}
\newcommand{\adhoc}{\mathcal{L}_{\textmd{AdHoc}}}
\newcommand{\lpoisson}{l_{\textmd{Poisson}}}
\newcommand{\lmc}{l_\textmd{Eff}}
\newcommand{\lbarlow}{\like_{\textmd{BB}}}
\newcommand{\hatmu}{\hat{\mu}}
\newcommand{\hatpoisson}{\hatmu_\textmd{Poisson}}
\newcommand{\hatmc}{\hatmu_\textmd{Eff}}
\newcommand{\au}{arb. unit}
\newcommand{\agpar}{\alpha}
\newcommand{\bgpar}{\beta}
\newcommand{\emcee}{\texttt{emcee}}
\newcommand{\meff}{m_\mathrm{Eff}}
\newcommand{\weff}{w_\mathrm{Eff}}


% This holds definitions of macros to enforce consistency in units.

% This file is the sole location for such definitions.  Check here to
% learn what there is and add new ones only here.

% also see defs.tex for names.


% see
%  http://ctan.org/pkg/siunitx
%  http://mirrors.ctan.org/macros/latex/contrib/siunitx/siunitx.pdf

% Examples:
%  % angles
%  \ang{1.5} off-axis
%
%  % just a unit
%  \si{\kilo\tonne}
%
%  % with a value:
%  \SI{10}{\mega\electronvolt}

%  range of values:
% \SIrange{60}{120}{\GeV}

% some shorthand notation
%\DeclareSIUnit \MBq {\mega\Bq}
\DeclareSIUnit \s {\second}
\DeclareSIUnit \ns {\nano\second}
\DeclareSIUnit \mus {\micro\second}
\DeclareSIUnit \ms {\milli\second}
\DeclareSIUnit \MB {\mega\byte}
\DeclareSIUnit \GB {\giga\byte}
\DeclareSIUnit \TB {\tera\byte}
\DeclareSIUnit \PB {\peta\byte}
\DeclareSIUnit \Mbps {\mega\bit/\s}
\DeclareSIUnit \Gbps {\giga\bit/\s}
\DeclareSIUnit \Tbps {\tera\bit/\s}
\DeclareSIUnit \Pbps {\peta\bit/\s}
\DeclareSIUnit \kton {\kilo\tonne} % changed  back to kton
\DeclareSIUnit \kt {\kilo\tonne}
\DeclareSIUnit \Mt {\mega\tonne}
\DeclareSIUnit \eV {\electronvolt}
\DeclareSIUnit \keV {\kilo\electronvolt}
\DeclareSIUnit \MeV {\mega\electronvolt}
\DeclareSIUnit \GeV {\giga\electronvolt}
\DeclareSIUnit \PeV {\peta\electronvolt}
\DeclareSIUnit \EeV {\exa\electronvolt}
\DeclareSIUnit \m {\meter}
\DeclareSIUnit \cm {\centi\meter}
\DeclareSIUnit \in {\inchcommand}
\DeclareSIUnit \km {\kilo\meter}
\DeclareSIUnit \kV {\kilo\volt}
\DeclareSIUnit \kW {\kilo\watt}
\DeclareSIUnit \MW {\mega\watt}
\DeclareSIUnit \MHz {\mega\hertz}
\DeclareSIUnit \mrad {\milli\radian}
\DeclareSIUnit \year {years}
\DeclareSIUnit \POT {POT}
\DeclareSIUnit \sig {$\sigma$}
\DeclareSIUnit\parsec{pc}
\DeclareSIUnit\lightyear{ly}
\DeclareSIUnit\foot{ft}
\DeclareSIUnit\ft{ft}
\DeclareSIUnit \ppb{ppb}
\DeclareSIUnit \ppt{ppt}
\DeclareSIUnit \samples{S}
\DeclareSIUnit \pe{PE}
\DeclareSIUnit \sr{\steradian}

\newcommand\SigmaOne{\SI{68.3}\percent}
\newcommand\SigmaTwo{\SI{95.4}\percent}

% "the Glashow Resonance energy"
\newcommand\GlashowEnergy{\SI{6.3}\PeV\xspace}
%

% The reconstructed deposited energy cut
\newcommand\EnergyCut{\SI{60}\TeV\xspace}

% The sample livetime
\newcommand\Livetime{\SI{7.5}\year\xspace}

% The segmented power law
\newcommand\minunfoldingenergy{1.995\times10^4}
\newcommand\maxunfoldingenergy{3.162\times10^8}
\newcommand\unfoldingsegments{13}

% Analysis parameters
\newcommand\astronorm{\Phi_\texttt{astro}}
\newcommand\astrodeltagamma{\gamma_\texttt{astro}}
\newcommand\convnorm{\Phi_\texttt{conv}}
\newcommand\promptnorm{\Phi_\texttt{prompt}}
\newcommand\pik{R_{K/\pi}}
\newcommand\atmonunubar{{2\nu/\left(\nu+\bar{\nu}\right)}_\texttt{atmo}}
\newcommand\crdeltagamma{\Delta\gamma_\texttt{CR}}
\newcommand\muonnorm{\Phi_\mu}
\newcommand\domeff{\epsilon_\texttt{DOM}}
\newcommand\holeice{\epsilon_\texttt{head-on}}
\newcommand\anisotropy{a_s}

% Parameters not in the analysis


\chapter{Results}

\section{Characterization of the astrophysical neutrino flux\label{sec:diffuse}}
\begingroup
\graphicspath{{results/HESE_Final_Paper/}}
\input{results/HESE_Final_Paper/sections/diffuse/diffuse}
\endgroup

\subsection{Generic models\label{sec:generic_models}}
\begingroup
\graphicspath{{results/HESE_Final_Paper/}}
\input{results/HESE_Final_Paper/sections/diffuse/generic_models}
\endgroup

\subsubsection{Single power-law flux\label{sec:spl}}
\begingroup
\graphicspath{{results/HESE_Final_Paper/}}
\input{results/HESE_Final_Paper/sections/diffuse/spl}
\endgroup

\subsubsection{Double power-law flux\label{sec:dpl}}
\begingroup
\graphicspath{{results/HESE_Final_Paper/}}
\input{results/HESE_Final_Paper/sections/diffuse/dpl}
\endgroup

\subsubsection{Single power law with spectral cutoff\label{sec:cutoff}}
\begingroup
\graphicspath{{results/HESE_Final_Paper/}}
\input{results/HESE_Final_Paper/sections/diffuse/cutoff}
\endgroup

\subsubsection{Log-parabola flux\label{sec:log_parabola}}
\begingroup
\graphicspath{{results/HESE_Final_Paper/}}
\input{results/HESE_Final_Paper/sections/diffuse/lppl}
\endgroup

\subsubsection{Segmented power-law flux\label{sec:unfolding}}
\begingroup
\graphicspath{{results/HESE_Final_Paper/}}
\input{results/HESE_Final_Paper/sections/diffuse/segmented}
\endgroup

\subsection{Atmospheric flux from charmed hadrons\label{sec:prompt}}
\begingroup
\graphicspath{{results/HESE_Final_Paper/}}
\input{results/HESE_Final_Paper/sections/diffuse/prompt}
\endgroup

\subsection{Source-specific models\label{sec:specific_models}}
\begingroup
\graphicspath{{results/HESE_Final_Paper/}}
\input{results/HESE_Final_Paper/sections/diffuse/specific_models}
\endgroup
\usepackage{tikz}
\usepackage{environ}

\usetikzlibrary{calc,trees,positioning,arrows,chains,shapes.geometric,%
    decorations.pathreplacing,decorations.pathmorphing,shapes,%
    matrix,shapes.symbols,backgrounds,fit} % required in the preamble
\usepackage{varwidth}

\makeatletter
\newsavebox{\measure@tikzpicture}
\NewEnviron{scaletikzpicturetowidth}[1]{%
  \def\tikz@width{#1}%
  \def\tikzscale{1}\begin{lrbox}{\measure@tikzpicture}%
  \BODY
  \end{lrbox}%
  \pgfmathparse{#1/\wd\measure@tikzpicture}%
  \edef\tikzscale{\pgfmathresult}%
  \BODY
}
\makeatother

\tikzset{
	%Define standard arrow tip
	>=stealth',
	%Define style for boxes
	box/.style={
		rectangle,
		rounded corners,
		dashed,
		draw=black, very thick,
		minimum height=2em,
		text centered,
		execute at begin node={\begin{varwidth}{28em}},
		execute at end node={\end{varwidth}}},
	solidbox/.style={
		rectangle,
		rounded corners,
		draw=black, very thick,
		minimum height=2em,
		text centered,
		execute at begin node={\begin{varwidth}{28em}},
			execute at end node={\end{varwidth}}},
    bigsolidbox/.style={
		rectangle,
		rounded corners,
		draw=black, very thick,
		minimum height=6cm,
		text centered,
		execute at begin node={\begin{varwidth}{28em}},
			execute at end node={\end{varwidth}}},
	% Define arrow style
	fw_arrow/.style={
		->,
		thick,
		shorten <=2pt,
		shorten >=2pt,},
	bw_arrow/.style={
		<-,
		thick,
		shorten <=2pt,
		shorten >=2pt,}
%	bigbox/.style={blue!50, thick, fill=blue!10, rounded corners, rectangle}
}

\definecolor{mc_gen_color}{RGB}{250,138,31}

\definecolor{det_sim_color}{RGB}{227,66,55}

\definecolor{llh_color}{RGB}{128,0,128}

\tikzstyle{bigboxGeneration} = [draw=mc_gen_color!50, thick, fill=mc_gen_color!20, rounded corners, rectangle]

\tikzstyle{bigboxDetector} = [draw=det_sim_color!50, thick, fill=det_sim_color!20, rounded corners, rectangle]

\tikzstyle{bigboxAnalysis} = [draw=llh_color!50, thick, fill=llh_color!20, rounded corners, rectangle]
   
\newcommand{\cmark}{\text{\ding{51}}}
\newcommand{\xmark}{\text{\ding{55}}}

\settitle{Precision measurements of the astrophysical neutrino flux}
\setauthor{Austin Schneider}
\setdepartment{Physics}
\doctors
\setgraddate{2020}
\setdefensedate{Some date in the near future}

\setfoca{Albrecht Karle}{Professor}{Physics}

\setabstract{The IceCube neutrino observatory has established the existence of a high-energy all-sky neutrino flux of astrophysical origin.
	This discovery was made using events interacting within a fiducial region of the detector surrounded by an active veto with reconstructed energy above \EnergyCut, commonly known as the high-energy starting event sample or HESE\@.
	We revisit the analysis of the HESE sample with an additional $\SI{4.5}\year$ of data, newer glacial ice models, and improved systematics treatment.
	This paper gives a detailed description of the sample, reports on the latest astrophysical neutrino flux measurements, and presents a source search for astrophysical neutrinos.
	We give the compatibility of these observations with specific isotropic flux models proposed in the literature as well as generic power-law-like scenarios.
	We find that the astrophysical neutrino spectrum, when assumed equal for neutrinos and anti-neutrinos and among neutrino flavors, is compatible with an unbroken power law, with a preferred spectral index of  $\SPLFreqBFIndex^{+\SPLFreqWilksUpperIndexDelta}_{-\SPLFreqWilksLowerIndexDelta}$ for the $\SigmaOne$ confidence interval.}
