\usepackage{uwthesis}
\usepackage{siunitx}

% This holds definitions of macros to enforce consistency in units.

% This file is the sole location for such definitions.  Check here to
% learn what there is and add new ones only here.

% also see defs.tex for names.


% see
%  http://ctan.org/pkg/siunitx
%  http://mirrors.ctan.org/macros/latex/contrib/siunitx/siunitx.pdf

% Examples:
%  % angles
%  \ang{1.5} off-axis
%
%  % just a unit
%  \si{\kilo\tonne}
%
%  % with a value:
%  \SI{10}{\mega\electronvolt}

%  range of values:
% \SIrange{60}{120}{\GeV}

% some shorthand notation
%\DeclareSIUnit \MBq {\mega\Bq}
\DeclareSIUnit \s {\second}
\DeclareSIUnit \ns {\nano\second}
\DeclareSIUnit \mus {\micro\second}
\DeclareSIUnit \ms {\milli\second}
\DeclareSIUnit \MB {\mega\byte}
\DeclareSIUnit \GB {\giga\byte}
\DeclareSIUnit \TB {\tera\byte}
\DeclareSIUnit \PB {\peta\byte}
\DeclareSIUnit \Mbps {\mega\bit/\s}
\DeclareSIUnit \Gbps {\giga\bit/\s}
\DeclareSIUnit \Tbps {\tera\bit/\s}
\DeclareSIUnit \Pbps {\peta\bit/\s}
\DeclareSIUnit \kton {\kilo\tonne} % changed  back to kton
\DeclareSIUnit \kt {\kilo\tonne}
\DeclareSIUnit \Mt {\mega\tonne}
\DeclareSIUnit \eV {\electronvolt}
\DeclareSIUnit \keV {\kilo\electronvolt}
\DeclareSIUnit \MeV {\mega\electronvolt}
\DeclareSIUnit \GeV {\giga\electronvolt}
\DeclareSIUnit \PeV {\peta\electronvolt}
\DeclareSIUnit \EeV {\exa\electronvolt}
\DeclareSIUnit \m {\meter}
\DeclareSIUnit \cm {\centi\meter}
\DeclareSIUnit \in {\inchcommand}
\DeclareSIUnit \km {\kilo\meter}
\DeclareSIUnit \kV {\kilo\volt}
\DeclareSIUnit \kW {\kilo\watt}
\DeclareSIUnit \MW {\mega\watt}
\DeclareSIUnit \MHz {\mega\hertz}
\DeclareSIUnit \mrad {\milli\radian}
\DeclareSIUnit \year {years}
\DeclareSIUnit \POT {POT}
\DeclareSIUnit \sig {$\sigma$}
\DeclareSIUnit\parsec{pc}
\DeclareSIUnit\lightyear{ly}
\DeclareSIUnit\foot{ft}
\DeclareSIUnit\ft{ft}
\DeclareSIUnit \ppb{ppb}
\DeclareSIUnit \ppt{ppt}
\DeclareSIUnit \samples{S}
\DeclareSIUnit \pe{PE}
\DeclareSIUnit \sr{\steradian}

\newcommand\SigmaOne{\SI{68.3}\percent}
\newcommand\SigmaTwo{\SI{95.4}\percent}

% "the Glashow Resonance energy"
\newcommand\GlashowEnergy{\SI{6.3}\PeV\xspace}
%

% The reconstructed deposited energy cut
\newcommand\EnergyCut{\SI{60}\TeV\xspace}

% The sample livetime
\newcommand\Livetime{\SI{7.5}\year\xspace}

% The segmented power law
\newcommand\minunfoldingenergy{1.995\times10^4}
\newcommand\maxunfoldingenergy{3.162\times10^8}
\newcommand\unfoldingsegments{13}

% Analysis parameters
\newcommand\astronorm{\Phi_\texttt{astro}}
\newcommand\astrodeltagamma{\gamma_\texttt{astro}}
\newcommand\convnorm{\Phi_\texttt{conv}}
\newcommand\promptnorm{\Phi_\texttt{prompt}}
\newcommand\pik{R_{K/\pi}}
\newcommand\atmonunubar{{2\nu/\left(\nu+\bar{\nu}\right)}_\texttt{atmo}}
\newcommand\crdeltagamma{\Delta\gamma_\texttt{CR}}
\newcommand\muonnorm{\Phi_\mu}
\newcommand\domeff{\epsilon_\texttt{DOM}}
\newcommand\holeice{\epsilon_\texttt{head-on}}
\newcommand\anisotropy{a_s}

% Parameters not in the analysis



\settitle{Precision measurements of the astrophysical neutrino flux}
\setauthor{Austin Schneider}
\setdepartment{Physics}
\doctors
\setgraddate{2020}
\setdefensedate{Some date in the near future}

\setfoca{Albrecht Karle}{Professor}{Physics}

\setabstract{The IceCube neutrino observatory has established the existence of a high-energy isotropic neutrino flux of astrophysical origin. This discovery was made using events interacting within a fiducial region of the detector surrounded by an active veto with reconstructed energy above $\SI{60}\TeV$, commonly known as the high-energy starting event sample or HESE. We perform a complete revisit of the HESE sample with an additional $\SI{4.5}\year$ of data, newer glacial ice models, and improved systematics treatment. This paper gives a detailed description of the sample, reports on the latest astrophysical neutrino flux measurements, and presents a source search for astrophysical neutrinos. We give the compatibility of these observations with detailed isotropic flux models proposed in the literature as well as generic power-law-like scenarios. We find that the astrophysical spectrum, when assumed equal for neutrinos and anti-neutrinos and among neutrino flavors, is compatible with an unbroken power law, with a preferred spectral index of $2.88^{+0.20}_{-0.19}$ for the $\SI{68.3}\percent$ confidence interval.}
