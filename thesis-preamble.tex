\usepackage{uwthesis}

\usepackage[caption=false]{subfig}
%\usepackage{subfigure}
\usepackage{graphicx}
\usepackage{bm}
\usepackage{amssymb}
\usepackage{amsmath}
\usepackage{amsfonts}
\usepackage[utf8]{inputenc}
\usepackage{placeins}
% defines table rules for professional looking tables
\usepackage{booktabs}
\usepackage{gensymb}
\usepackage[colorlinks=true,allcolors=blue]{hyperref}
\usepackage{microtype}
\usepackage{multirow, makecell}
\usepackage[capitalise]{cleveref}
\usepackage[utf8]{inputenc}
\usepackage{listings}
\usepackage{siunitx}
%\sisetup{detect-weight=true, detect-family=true}
\sisetup{detect-all = true}
\usepackage{xspace}
\usepackage{comment}
\usepackage{ragged2e}
\usepackage{lineno}

\newcommand{\Python}{\texttt{Python}}
\newcommand{\MCEq}{\texttt{MCE{\scriptsize Q}}}
\newcommand{\MMC}{\texttt{MMC}}
\newcommand{\PROPOSAL}{\texttt{PROPOSAL}}
\newcommand{\CORSIKA}{\texttt{CORSIKA}}
\newcommand{\MUONGUN}{\texttt{MUONGUN}}
\newcommand{\PHOTOSPLINE}{\texttt{PHOTOSPLINE}}
\newcommand{\like}{\mathcal{L}}
\newcommand{\likeSAY}{\mathcal{L}_{\textrm{Eff}}}
\newcommand{\pdfSxi}{S(\vec{x}_i)}
\newcommand{\TS}{\mathrm{TS}}
\newcommand{\Fermi}{\textit{Fermi}}
\newcommand{\nuveto}{{\LARGE $\nu$}\texttt{eto}}

\DeclareMathOperator*{\argmax}{arg\,max}
\DeclareMathOperator*{\argmin}{arg\,min}

\newcommand{\refeq}[1]{Eq.~(\ref{#1})}
\newcommand{\refeqs}[2]{Eqs.~(\ref{#1})~and~(\ref{#2})}
\newcommand{\refeqss}[3]{Eqs.~(\ref{#1}), (\ref{#2})~and~(\ref{#3})}
\newcommand{\reffig}[1]{Fig.~\ref{#1}}
\newcommand{\reffigs}[2]{Figs.~\ref{#1}~and~\ref{#2}}
\newcommand{\refsec}[1]{Section~\ref{#1}}
\newcommand{\refsecs}[5]{Sections~\ref{#1},~\ref{#2},~\ref{#3},~\ref{#4},~and~\ref{#5}}
\newcommand{\refappsec}[1]{Appendix Section~\ref{#1}}
\newcommand{\refapp}[1]{Appendix~\ref{#1}}
\newcommand{\reftab}[1]{Table~\ref{#1}}
\newcommand{\refref}[1]{Ref.~\cite{#1}}
\newcommand{\refrefs}[2]{Refs.~\cite{#1}~and~\cite{#2}}

% This holds definitions of macros to enforce consistency in units.

% This file is the sole location for such definitions.  Check here to
% learn what there is and add new ones only here.

% also see defs.tex for names.


% see
%  http://ctan.org/pkg/siunitx
%  http://mirrors.ctan.org/macros/latex/contrib/siunitx/siunitx.pdf

% Examples:
%  % angles
%  \ang{1.5} off-axis
%
%  % just a unit
%  \si{\kilo\tonne}
%
%  % with a value:
%  \SI{10}{\mega\electronvolt}

%  range of values:
% \SIrange{60}{120}{\GeV}

% some shorthand notation
%\DeclareSIUnit \MBq {\mega\Bq}
\DeclareSIUnit \s {\second}
\DeclareSIUnit \ns {\nano\second}
\DeclareSIUnit \mus {\micro\second}
\DeclareSIUnit \ms {\milli\second}
\DeclareSIUnit \MB {\mega\byte}
\DeclareSIUnit \GB {\giga\byte}
\DeclareSIUnit \TB {\tera\byte}
\DeclareSIUnit \PB {\peta\byte}
\DeclareSIUnit \Mbps {\mega\bit/\s}
\DeclareSIUnit \Gbps {\giga\bit/\s}
\DeclareSIUnit \Tbps {\tera\bit/\s}
\DeclareSIUnit \Pbps {\peta\bit/\s}
\DeclareSIUnit \kton {\kilo\tonne} % changed  back to kton
\DeclareSIUnit \kt {\kilo\tonne}
\DeclareSIUnit \Mt {\mega\tonne}
\DeclareSIUnit \eV {\electronvolt}
\DeclareSIUnit \keV {\kilo\electronvolt}
\DeclareSIUnit \MeV {\mega\electronvolt}
\DeclareSIUnit \GeV {\giga\electronvolt}
\DeclareSIUnit \PeV {\peta\electronvolt}
\DeclareSIUnit \EeV {\exa\electronvolt}
\DeclareSIUnit \m {\meter}
\DeclareSIUnit \cm {\centi\meter}
\DeclareSIUnit \in {\inchcommand}
\DeclareSIUnit \km {\kilo\meter}
\DeclareSIUnit \kV {\kilo\volt}
\DeclareSIUnit \kW {\kilo\watt}
\DeclareSIUnit \MW {\mega\watt}
\DeclareSIUnit \MHz {\mega\hertz}
\DeclareSIUnit \mrad {\milli\radian}
\DeclareSIUnit \year {years}
\DeclareSIUnit \POT {POT}
\DeclareSIUnit \sig {$\sigma$}
\DeclareSIUnit\parsec{pc}
\DeclareSIUnit\lightyear{ly}
\DeclareSIUnit\foot{ft}
\DeclareSIUnit\ft{ft}
\DeclareSIUnit \ppb{ppb}
\DeclareSIUnit \ppt{ppt}
\DeclareSIUnit \samples{S}
\DeclareSIUnit \pe{PE}
\DeclareSIUnit \sr{\steradian}

\newcommand\SigmaOne{\SI{68.3}\percent}
\newcommand\SigmaTwo{\SI{95.4}\percent}

% "the Glashow Resonance energy"
\newcommand\GlashowEnergy{\SI{6.3}\PeV\xspace}
%

% The reconstructed deposited energy cut
\newcommand\EnergyCut{\SI{60}\TeV\xspace}

% The sample livetime
\newcommand\Livetime{\SI{7.5}\year\xspace}

% The segmented power law
\newcommand\minunfoldingenergy{1.995\times10^4}
\newcommand\maxunfoldingenergy{3.162\times10^8}
\newcommand\unfoldingsegments{13}

% Analysis parameters
\newcommand\astronorm{\Phi_\texttt{astro}}
\newcommand\astrodeltagamma{\gamma_\texttt{astro}}
\newcommand\convnorm{\Phi_\texttt{conv}}
\newcommand\promptnorm{\Phi_\texttt{prompt}}
\newcommand\pik{R_{K/\pi}}
\newcommand\atmonunubar{{2\nu/\left(\nu+\bar{\nu}\right)}_\texttt{atmo}}
\newcommand\crdeltagamma{\Delta\gamma_\texttt{CR}}
\newcommand\muonnorm{\Phi_\mu}
\newcommand\domeff{\epsilon_\texttt{DOM}}
\newcommand\holeice{\epsilon_\texttt{head-on}}
\newcommand\anisotropy{a_s}

% Parameters not in the analysis



\usepackage{xparse}

\ExplSyntaxOn
\DeclareExpandableDocumentCommand{\eval}{m}{\fp_eval:n {#1}}
\ExplSyntaxOff


% SPL Frequentist Wilks results
\newcommand\SPLFreqBFCRDeltaGamma{-0.053}
\newcommand\SPLFreqWilksUpperCRDeltaGamma{-0.006}
\newcommand\SPLFreqWilksLowerCRDeltaGamma{-0.184}
\newcommand\SPLFreqWilksUpperCRDeltaGammaDelta{0.048}
\newcommand\SPLFreqWilksLowerCRDeltaGammaDelta{0.131}
\newcommand\SPLFreqWilksCRDeltaGammaSummary{-0.053^{+0.048}_{-0.131}}
\newcommand\SPLFreqBFAtmoNuRatio{1.002}
\newcommand\SPLFreqWilksUpperAtmoNuRatio{1.102}
\newcommand\SPLFreqWilksLowerAtmoNuRatio{0.902}
\newcommand\SPLFreqWilksUpperAtmoNuRatioDelta{0.100}
\newcommand\SPLFreqWilksLowerAtmoNuRatioDelta{0.100}
\newcommand\SPLFreqWilksAtmoNuRatioSummary{1.002^{+0.100}_{-0.100}}
\newcommand\SPLFreqBFAniScale{1.00}
\newcommand\SPLFreqWilksUpperAniScale{1.20}
\newcommand\SPLFreqWilksLowerAniScale{0.80}
\newcommand\SPLFreqWilksUpperAniScaleDelta{0.20}
\newcommand\SPLFreqWilksLowerAniScaleDelta{0.20}
\newcommand\SPLFreqWilksAniScaleSummary{1.00^{+0.20}_{-0.20}}
\newcommand\SPLFreqBFIndex{2.88}
\newcommand\SPLFreqWilksUpperIndex{3.08}
\newcommand\SPLFreqWilksLowerIndex{2.69}
\newcommand\SPLFreqWilksUpperIndexDelta{0.20}
\newcommand\SPLFreqWilksLowerIndexDelta{0.19}
\newcommand\SPLFreqWilksIndexSummary{2.88^{+0.20}_{-0.19}}
\newcommand\SPLFreqBFNorm{6.44}
\newcommand\SPLFreqWilksUpperNorm{7.92}
\newcommand\SPLFreqWilksLowerNorm{4.85}
\newcommand\SPLFreqWilksUpperNormDelta{1.48}
\newcommand\SPLFreqWilksLowerNormDelta{1.60}
\newcommand\SPLFreqWilksNormSummary{6.44^{+1.48}_{-1.60}}
\newcommand\SPLFreqBFConvNorm{1.01}
\newcommand\SPLFreqWilksUpperConvNorm{1.36}
\newcommand\SPLFreqWilksLowerConvNorm{0.67}
\newcommand\SPLFreqWilksUpperConvNormDelta{0.35}
\newcommand\SPLFreqWilksLowerConvNormDelta{0.33}
\newcommand\SPLFreqWilksConvNormSummary{1.01^{+0.35}_{-0.33}}
\newcommand\SPLFreqBFDOMEff{0.951}
\newcommand\SPLFreqWilksUpperDOMEff{1.041}
\newcommand\SPLFreqWilksLowerDOMEff{0.884}
\newcommand\SPLFreqWilksUpperDOMEffDelta{0.090}
\newcommand\SPLFreqWilksLowerDOMEffDelta{0.067}
\newcommand\SPLFreqWilksDOMEffSummary{0.951^{+0.090}_{-0.067}}
\newcommand\SPLFreqBFHoleIce{-0.06}
\newcommand\SPLFreqWilksUpperHoleIce{0.45}
\newcommand\SPLFreqWilksLowerHoleIce{-0.54}
\newcommand\SPLFreqWilksUpperHoleIceDelta{0.51}
\newcommand\SPLFreqWilksLowerHoleIceDelta{0.48}
\newcommand\SPLFreqWilksHoleIceSummary{-0.06^{+0.51}_{-0.48}}
\newcommand\SPLFreqBFMuonNorm{1.21}
\newcommand\SPLFreqWilksUpperMuonNorm{1.67}
\newcommand\SPLFreqWilksLowerMuonNorm{0.75}
\newcommand\SPLFreqWilksUpperMuonNormDelta{0.46}
\newcommand\SPLFreqWilksLowerMuonNormDelta{0.46}
\newcommand\SPLFreqWilksMuonNormSummary{1.21^{+0.46}_{-0.46}}
\newcommand\SPLFreqBFKPi{1.000}
\newcommand\SPLFreqWilksUpperKPi{1.100}
\newcommand\SPLFreqWilksLowerKPi{0.901}
\newcommand\SPLFreqWilksUpperKPiDelta{0.100}
\newcommand\SPLFreqWilksLowerKPiDelta{0.099}
\newcommand\SPLFreqWilksKPiSummary{1.000^{+0.100}_{-0.099}}
\newcommand\SPLFreqBFPromptNorm{0.00}
\newcommand\SPLFreqWilksUpperPromptNorm{5.17}
\newcommand\SPLFreqWilksLowerPromptNorm{0.00}
\newcommand\SPLFreqWilksUpperPromptNormDelta{5.17}
\newcommand\SPLFreqWilksLowerPromptNormDelta{0.00}
\newcommand\SPLFreqWilksPromptNormSummary{0.00^{+5.17}_{-0.00}}

% prompt normalization
\newcommand\NoAstroPromptNorm{25.7} %25.65
\newcommand\ICFiftyNinePromptUpperLimit{3.80} % ERS at 90 % CL
\newcommand\SPLFreqWilksNinetyUpperLimit{9.65}

% prompt bayes factor
\newcommand\BayesFactorSubstantialPromptNorm{8.65}
\newcommand\BayesFactorStrongPromptNorm{13.21}
\newcommand\BayesFactorVeryStrongPromptNorm{17.18}
\newcommand\BayesFactorDecisivePromptNorm{21.16}

% SPL Bayesian HPD results

\newcommand\SPLBayesMAPCRDeltaGamma{-0.039}
\newcommand\SPLBayesHPDUpperCRDeltaGamma{0.009}
\newcommand\SPLBayesHPDLowerCRDeltaGamma{-0.086}
\newcommand\SPLBayesHPDUpperCRDeltaGammaDelta{0.048}
\newcommand\SPLBayesHPDLowerCRDeltaGammaDelta{0.047}
\newcommand\SPLBayesCRDeltaGammaSummary{-0.039^{+0.048}_{-0.047}}\newcommand\SPLBayesMAPAtmoNuRatio{1.010}
\newcommand\SPLBayesHPDUpperAtmoNuRatio{1.101}
\newcommand\SPLBayesHPDLowerAtmoNuRatio{0.906}
\newcommand\SPLBayesHPDUpperAtmoNuRatioDelta{0.091}
\newcommand\SPLBayesHPDLowerAtmoNuRatioDelta{0.105}
\newcommand\SPLBayesAtmoNuRatioSummary{1.010^{+0.091}_{-0.105}}\newcommand\SPLBayesMAPAniScale{0.97}
\newcommand\SPLBayesHPDUpperAniScale{1.19}
\newcommand\SPLBayesHPDLowerAniScale{0.79}
\newcommand\SPLBayesHPDUpperAniScaleDelta{0.22}
\newcommand\SPLBayesHPDLowerAniScaleDelta{0.18}
\newcommand\SPLBayesAniScaleSummary{0.97^{+0.22}_{-0.18}}\newcommand\SPLBayesMAPIndex{2.85}
\newcommand\SPLBayesHPDUpperIndex{3.10}
\newcommand\SPLBayesHPDLowerIndex{2.68}
\newcommand\SPLBayesHPDUpperIndexDelta{0.26}
\newcommand\SPLBayesHPDLowerIndexDelta{0.17}
\newcommand\SPLBayesIndexSummary{2.85^{+0.26}_{-0.17}}\newcommand\SPLBayesMAPNorm{5.63}
\newcommand\SPLBayesHPDUpperNorm{7.14}
\newcommand\SPLBayesHPDLowerNorm{4.10}
\newcommand\SPLBayesHPDUpperNormDelta{1.52}
\newcommand\SPLBayesHPDLowerNormDelta{1.52}
\newcommand\SPLBayesNormSummary{5.63^{+1.52}_{-1.52}}\newcommand\SPLBayesMAPConvNorm{0.98}
\newcommand\SPLBayesHPDUpperConvNorm{1.28}
\newcommand\SPLBayesHPDLowerConvNorm{0.58}
\newcommand\SPLBayesHPDUpperConvNormDelta{0.31}
\newcommand\SPLBayesHPDLowerConvNormDelta{0.40}
\newcommand\SPLBayesConvNormSummary{0.98^{+0.31}_{-0.40}}\newcommand\SPLBayesMAPDOMEff{0.933}
\newcommand\SPLBayesHPDUpperDOMEff{1.007}
\newcommand\SPLBayesHPDLowerDOMEff{0.853}
\newcommand\SPLBayesHPDUpperDOMEffDelta{0.074}
\newcommand\SPLBayesHPDLowerDOMEffDelta{0.080}
\newcommand\SPLBayesDOMEffSummary{0.933^{+0.074}_{-0.080}}\newcommand\SPLBayesMAPHoleIce{-0.05}
\newcommand\SPLBayesHPDUpperHoleIce{0.44}
\newcommand\SPLBayesHPDLowerHoleIce{-0.61}
\newcommand\SPLBayesHPDUpperHoleIceDelta{0.49}
\newcommand\SPLBayesHPDLowerHoleIceDelta{0.56}
\newcommand\SPLBayesHoleIceSummary{-0.05^{+0.49}_{-0.56}}\newcommand\SPLBayesMAPMuonNorm{1.17}
\newcommand\SPLBayesHPDUpperMuonNorm{1.67}
\newcommand\SPLBayesHPDLowerMuonNorm{0.75}
\newcommand\SPLBayesHPDUpperMuonNormDelta{0.50}
\newcommand\SPLBayesHPDLowerMuonNormDelta{0.42}
\newcommand\SPLBayesMuonNormSummary{1.17^{+0.50}_{-0.42}}\newcommand\SPLBayesMAPKPi{1.011}
\newcommand\SPLBayesHPDUpperKPi{1.097}
\newcommand\SPLBayesHPDLowerKPi{0.897}
\newcommand\SPLBayesHPDUpperKPiDelta{0.086}
\newcommand\SPLBayesHPDLowerKPiDelta{0.114}
\newcommand\SPLBayesKPiSummary{1.011^{+0.086}_{-0.114}}\newcommand\SPLBayesMAPPromptNorm{0.18}
\newcommand\SPLBayesHPDUpperPromptNorm{5.97}
\newcommand\SPLBayesHPDLowerPromptNorm{0.00}
\newcommand\SPLBayesHPDUpperPromptNormDelta{5.79}
\newcommand\SPLBayesHPDLowerPromptNormDelta{0.18}
\newcommand\SPLBayesPromptNormSummary{0.18^{+5.79}_{-0.18}}

\newcommand\SPL{Single Power Law}
\newcommand\SPLBayes{0.0872}
\newcommand\SPLSPLBayes{2.53\times10^{63}}
\newcommand\SPLSPLPhiLower{-}
\newcommand\SPLSPLPhiMode{-}
\newcommand\SPLSPLPhiUpper{-}
\newcommand\SPLSPLPhiSummary{-}
\newcommand\SPLSPLGammaLower{-}
\newcommand\SPLSPLGammaMode{-}
\newcommand\SPLSPLGammaUpper{-}
\newcommand\SPLSPLGammaSummary{-}
\newcommand\SPLTableSummary{Generic & \SPL & $\SPLBayes$ & $\SPLSPLBayes$ & $\SPLSPLGammaSummary$ & $\SPLSPLPhiSummary$}
\newcommand\DPL{Double Power Law}
\newcommand\DPLBayes{1.93\times10^{-3}}
\newcommand\DPLSPLBayes{2.53\times10^{63}}
\newcommand\DPLSPLPhiLower{-}
\newcommand\DPLSPLPhiMode{-}
\newcommand\DPLSPLPhiUpper{-}
\newcommand\DPLSPLPhiSummary{-}
\newcommand\DPLSPLGammaLower{-}
\newcommand\DPLSPLGammaMode{-}
\newcommand\DPLSPLGammaUpper{-}
\newcommand\DPLSPLGammaSummary{-}
\newcommand\DPLTableSummary{Generic & \DPL & $\DPLBayes$ & $\DPLSPLBayes$ & $\DPLSPLGammaSummary$ & $\DPLSPLPhiSummary$}
\newcommand\LP{Log Parabola}
\newcommand\LPBayes{2.05\times10^{-3}}
\newcommand\LPSPLBayes{2.53\times10^{63}}
\newcommand\LPSPLPhiLower{-}
\newcommand\LPSPLPhiMode{-}
\newcommand\LPSPLPhiUpper{-}
\newcommand\LPSPLPhiSummary{-}
\newcommand\LPSPLGammaLower{-}
\newcommand\LPSPLGammaMode{-}
\newcommand\LPSPLGammaUpper{-}
\newcommand\LPSPLGammaSummary{-}
\newcommand\LPTableSummary{Generic & \LP & $\LPBayes$ & $\LPSPLBayes$ & $\LPSPLGammaSummary$ & $\LPSPLPhiSummary$}
\newcommand\Stecker{Stecker~\cite{Stecker:2013fxa}}
\newcommand\SteckerBayes{4.32\times10^{-13}}
\newcommand\SteckerSPLBayes{1.45\times10^{-10}}
\newcommand\SteckerSPLPhiLower{2.95}
\newcommand\SteckerSPLPhiMode{4.08}
\newcommand\SteckerSPLPhiUpper{5.88}
\newcommand\SteckerSPLPhiLowerDelta{\eval{\SteckerSPLPhiMode - \SteckerSPLPhiLower}}
\newcommand\SteckerSPLPhiUpperDelta{\eval{\SteckerSPLPhiUpper - \SteckerSPLPhiMode}}
\newcommand\SteckerSPLPhiSummary{\SteckerSPLPhiMode^{+\SteckerSPLPhiUpperDelta}_{-\SteckerSPLPhiLowerDelta}}
\newcommand\SteckerSPLGammaLower{3.5}
\newcommand\SteckerSPLGammaMode{3.97}
\newcommand\SteckerSPLGammaUpper{4.51}
\newcommand\SteckerSPLGammaLowerDelta{\eval{\SteckerSPLGammaMode - \SteckerSPLGammaLower}}
\newcommand\SteckerSPLGammaUpperDelta{\eval{\SteckerSPLGammaUpper - \SteckerSPLGammaMode}}
\newcommand\SteckerSPLGammaSummary{\SteckerSPLGammaMode^{+\SteckerSPLGammaUpperDelta}_{-\SteckerSPLGammaLowerDelta}}
\newcommand\SteckerTableSummary{AGN core & \Stecker & $\SteckerBayes$ & $\SteckerSPLBayes$ & $\SteckerSPLGammaSummary$ & $\SteckerSPLPhiSummary$}
\newcommand\Fang{Fang et al.~\cite{Fang:2017zjf}}
\newcommand\FangBayes{0.281}
\newcommand\FangSPLBayes{0.248}
\newcommand\FangSPLPhiLower{1.12}
\newcommand\FangSPLPhiMode{2.56}
\newcommand\FangSPLPhiUpper{3.84}
\newcommand\FangSPLPhiLowerDelta{\eval{\FangSPLPhiMode - \FangSPLPhiLower}}
\newcommand\FangSPLPhiUpperDelta{\eval{\FangSPLPhiUpper - \FangSPLPhiMode}}
\newcommand\FangSPLPhiSummary{\FangSPLPhiMode^{+\FangSPLPhiUpperDelta}_{-\FangSPLPhiLowerDelta}}
\newcommand\FangSPLGammaLower{3.33}
\newcommand\FangSPLGammaMode{3.83}
\newcommand\FangSPLGammaUpper{4.64}
\newcommand\FangSPLGammaLowerDelta{\eval{\FangSPLGammaMode - \FangSPLGammaLower}}
\newcommand\FangSPLGammaUpperDelta{\eval{\FangSPLGammaUpper - \FangSPLGammaMode}}
\newcommand\FangSPLGammaSummary{\FangSPLGammaMode^{+\FangSPLGammaUpperDelta}_{-\FangSPLGammaLowerDelta}}
\newcommand\FangTableSummary{AGN & \Fang & $\FangBayes$ & $\FangSPLBayes$ & $\FangSPLGammaSummary$ & $\FangSPLPhiSummary$}
\newcommand\KimuraBOne{Kimura et al. (B1)~\cite{Kimura:2014jba}}
\newcommand\KimuraBOneBayes{4.84\times10^{-6}}
\newcommand\KimuraBOneSPLBayes{8.38\times10^{-7}}
\newcommand\KimuraBOneSPLPhiLower{0.0}
\newcommand\KimuraBOneSPLPhiMode{0.98}
\newcommand\KimuraBOneSPLPhiUpper{2.02}
\newcommand\KimuraBOneSPLPhiLowerDelta{\eval{\KimuraBOneSPLPhiMode - \KimuraBOneSPLPhiLower}}
\newcommand\KimuraBOneSPLPhiUpperDelta{\eval{\KimuraBOneSPLPhiUpper - \KimuraBOneSPLPhiMode}}
\newcommand\KimuraBOneSPLPhiSummary{\KimuraBOneSPLPhiMode^{+\KimuraBOneSPLPhiUpperDelta}_{-\KimuraBOneSPLPhiLowerDelta}}
\newcommand\KimuraBOneSPLGammaLower{3.83}
\newcommand\KimuraBOneSPLGammaMode{4.5}
\newcommand\KimuraBOneSPLGammaUpper{5.0}
\newcommand\KimuraBOneSPLGammaLowerDelta{\eval{\KimuraBOneSPLGammaMode - \KimuraBOneSPLGammaLower}}
\newcommand\KimuraBOneSPLGammaUpperDelta{\eval{\KimuraBOneSPLGammaUpper - \KimuraBOneSPLGammaMode}}
\newcommand\KimuraBOneSPLGammaSummary{\KimuraBOneSPLGammaMode^{+\KimuraBOneSPLGammaUpperDelta}_{-\KimuraBOneSPLGammaLowerDelta}}
\newcommand\KimuraBOneTableSummary{LLAGN & \KimuraBOne & $\KimuraBOneBayes$ & $\KimuraBOneSPLBayes$ & $\KimuraBOneSPLGammaSummary$ & $\KimuraBOneSPLPhiSummary$}
\newcommand\KimuraBFour{Kimura et al. (B4)~\cite{Kimura:2014jba}}
\newcommand\KimuraBFourBayes{3.44\times10^{-4}}
\newcommand\KimuraBFourSPLBayes{0.666}
\newcommand\KimuraBFourSPLPhiLower{0.62}
\newcommand\KimuraBFourSPLPhiMode{1.39}
\newcommand\KimuraBFourSPLPhiUpper{2.57}
\newcommand\KimuraBFourSPLPhiLowerDelta{\eval{\KimuraBFourSPLPhiMode - \KimuraBFourSPLPhiLower}}
\newcommand\KimuraBFourSPLPhiUpperDelta{\eval{\KimuraBFourSPLPhiUpper - \KimuraBFourSPLPhiMode}}
\newcommand\KimuraBFourSPLPhiSummary{\KimuraBFourSPLPhiMode^{+\KimuraBFourSPLPhiUpperDelta}_{-\KimuraBFourSPLPhiLowerDelta}}
\newcommand\KimuraBFourSPLGammaLower{2.17}
\newcommand\KimuraBFourSPLGammaMode{2.43}
\newcommand\KimuraBFourSPLGammaUpper{2.74}
\newcommand\KimuraBFourSPLGammaLowerDelta{\eval{\KimuraBFourSPLGammaMode - \KimuraBFourSPLGammaLower}}
\newcommand\KimuraBFourSPLGammaUpperDelta{\eval{\KimuraBFourSPLGammaUpper - \KimuraBFourSPLGammaMode}}
\newcommand\KimuraBFourSPLGammaSummary{\KimuraBFourSPLGammaMode^{+\KimuraBFourSPLGammaUpperDelta}_{-\KimuraBFourSPLGammaLowerDelta}}
\newcommand\KimuraBFourTableSummary{LLAGN & \KimuraBFour & $\KimuraBFourBayes$ & $\KimuraBFourSPLBayes$ & $\KimuraBFourSPLGammaSummary$ & $\KimuraBFourSPLPhiSummary$}
\newcommand\KimuraTwoComp{Kimura et al. (two component)~\cite{Kimura:2014jba}}
\newcommand\KimuraTwoCompBayes{1.73\times10^{-4}}
\newcommand\KimuraTwoCompSPLBayes{6.12\times10^{-6}}
\newcommand\KimuraTwoCompSPLPhiLower{0.0}
\newcommand\KimuraTwoCompSPLPhiMode{0.0}
\newcommand\KimuraTwoCompSPLPhiUpper{0.69}
\newcommand\KimuraTwoCompSPLPhiLowerDelta{\eval{\KimuraTwoCompSPLPhiMode - \KimuraTwoCompSPLPhiLower}}
\newcommand\KimuraTwoCompSPLPhiUpperDelta{\eval{\KimuraTwoCompSPLPhiUpper - \KimuraTwoCompSPLPhiMode}}
\newcommand\KimuraTwoCompSPLPhiSummary{\KimuraTwoCompSPLPhiMode^{+\KimuraTwoCompSPLPhiUpperDelta}_{-\KimuraTwoCompSPLPhiLowerDelta}}
\newcommand\KimuraTwoCompSPLGammaLower{3.42}
\newcommand\KimuraTwoCompSPLGammaMode{4.15}
\newcommand\KimuraTwoCompSPLGammaUpper{4.99}
\newcommand\KimuraTwoCompSPLGammaLowerDelta{\eval{\KimuraTwoCompSPLGammaMode - \KimuraTwoCompSPLGammaLower}}
\newcommand\KimuraTwoCompSPLGammaUpperDelta{\eval{\KimuraTwoCompSPLGammaUpper - \KimuraTwoCompSPLGammaMode}}
\newcommand\KimuraTwoCompSPLGammaSummary{\KimuraTwoCompSPLGammaMode^{+\KimuraTwoCompSPLGammaUpperDelta}_{-\KimuraTwoCompSPLGammaLowerDelta}}
\newcommand\KimuraTwoCompTableSummary{LLAGN & \KimuraTwoComp & $\KimuraTwoCompBayes$ & $\KimuraTwoCompSPLBayes$ & $\KimuraTwoCompSPLGammaSummary$ & $\KimuraTwoCompSPLPhiSummary$}
\newcommand\MariaBLLacs{Padovani et al.~\cite{Padovani:2015mba}}
\newcommand\MariaBLLacsBayes{6.20\times10^{-11}}
\newcommand\MariaBLLacsSPLBayes{3.32\times10^{-7}}
\newcommand\MariaBLLacsSPLPhiLower{3.51}
\newcommand\MariaBLLacsSPLPhiMode{4.97}
\newcommand\MariaBLLacsSPLPhiUpper{6.65}
\newcommand\MariaBLLacsSPLPhiLowerDelta{\eval{\MariaBLLacsSPLPhiMode - \MariaBLLacsSPLPhiLower}}
\newcommand\MariaBLLacsSPLPhiUpperDelta{\eval{\MariaBLLacsSPLPhiUpper - \MariaBLLacsSPLPhiMode}}
\newcommand\MariaBLLacsSPLPhiSummary{\MariaBLLacsSPLPhiMode^{+\MariaBLLacsSPLPhiUpperDelta}_{-\MariaBLLacsSPLPhiLowerDelta}}
\newcommand\MariaBLLacsSPLGammaLower{3.25}
\newcommand\MariaBLLacsSPLGammaMode{3.59}
\newcommand\MariaBLLacsSPLGammaUpper{4.18}
\newcommand\MariaBLLacsSPLGammaLowerDelta{\eval{\MariaBLLacsSPLGammaMode - \MariaBLLacsSPLGammaLower}}
\newcommand\MariaBLLacsSPLGammaUpperDelta{\eval{\MariaBLLacsSPLGammaUpper - \MariaBLLacsSPLGammaMode}}
\newcommand\MariaBLLacsSPLGammaSummary{\MariaBLLacsSPLGammaMode^{+\MariaBLLacsSPLGammaUpperDelta}_{-\MariaBLLacsSPLGammaLowerDelta}}
\newcommand\MariaBLLacsTableSummary{BLLac & \MariaBLLacs & $\MariaBLLacsBayes$ & $\MariaBLLacsSPLBayes$ & $\MariaBLLacsSPLGammaSummary$ & $\MariaBLLacsSPLPhiSummary$}
\newcommand\MurasechockedJets{Senno et al.~\cite{Senno:2015tsn}}
\newcommand\MurasechockedJetsBayes{0.256}
\newcommand\MurasechockedJetsSPLBayes{3.52}
\newcommand\MurasechockedJetsSPLPhiLower{2.02}
\newcommand\MurasechockedJetsSPLPhiMode{3.36}
\newcommand\MurasechockedJetsSPLPhiUpper{4.92}
\newcommand\MurasechockedJetsSPLPhiLowerDelta{\eval{\MurasechockedJetsSPLPhiMode - \MurasechockedJetsSPLPhiLower}}
\newcommand\MurasechockedJetsSPLPhiUpperDelta{\eval{\MurasechockedJetsSPLPhiUpper - \MurasechockedJetsSPLPhiMode}}
\newcommand\MurasechockedJetsSPLPhiSummary{\MurasechockedJetsSPLPhiMode^{+\MurasechockedJetsSPLPhiUpperDelta}_{-\MurasechockedJetsSPLPhiLowerDelta}}
\newcommand\MurasechockedJetsSPLGammaLower{3.05}
\newcommand\MurasechockedJetsSPLGammaMode{3.67}
\newcommand\MurasechockedJetsSPLGammaUpper{4.24}
\newcommand\MurasechockedJetsSPLGammaLowerDelta{\eval{\MurasechockedJetsSPLGammaMode - \MurasechockedJetsSPLGammaLower}}
\newcommand\MurasechockedJetsSPLGammaUpperDelta{\eval{\MurasechockedJetsSPLGammaUpper - \MurasechockedJetsSPLGammaMode}}
\newcommand\MurasechockedJetsSPLGammaSummary{\MurasechockedJetsSPLGammaMode^{+\MurasechockedJetsSPLGammaUpperDelta}_{-\MurasechockedJetsSPLGammaLowerDelta}}
\newcommand\MurasechockedJetsTableSummary{GRB Choked Jet & \MurasechockedJets & $\MurasechockedJetsBayes$ & $\MurasechockedJetsSPLBayes$ & $\MurasechockedJetsSPLGammaSummary$ & $\MurasechockedJetsSPLPhiSummary$}
\newcommand\SBGminBmodel{Bartos et al.~\cite{Bartos:2015xpa}}
\newcommand\SBGminBmodelBayes{1.15\times10^{-14}}
\newcommand\SBGminBmodelSPLBayes{2.81\times10^{-16}}
\newcommand\SBGminBmodelSPLPhiLower{0.0}
\newcommand\SBGminBmodelSPLPhiMode{0.0}
\newcommand\SBGminBmodelSPLPhiUpper{0.49}
\newcommand\SBGminBmodelSPLPhiLowerDelta{\eval{\SBGminBmodelSPLPhiMode - \SBGminBmodelSPLPhiLower}}
\newcommand\SBGminBmodelSPLPhiUpperDelta{\eval{\SBGminBmodelSPLPhiUpper - \SBGminBmodelSPLPhiMode}}
\newcommand\SBGminBmodelSPLPhiSummary{\SBGminBmodelSPLPhiMode^{+\SBGminBmodelSPLPhiUpperDelta}_{-\SBGminBmodelSPLPhiLowerDelta}}
\newcommand\SBGminBmodelSPLGammaLower{3.42}
\newcommand\SBGminBmodelSPLGammaMode{4.25}
\newcommand\SBGminBmodelSPLGammaUpper{5.0}
\newcommand\SBGminBmodelSPLGammaLowerDelta{\eval{\SBGminBmodelSPLGammaMode - \SBGminBmodelSPLGammaLower}}
\newcommand\SBGminBmodelSPLGammaUpperDelta{\eval{\SBGminBmodelSPLGammaUpper - \SBGminBmodelSPLGammaMode}}
\newcommand\SBGminBmodelSPLGammaSummary{\SBGminBmodelSPLGammaMode^{+\SBGminBmodelSPLGammaUpperDelta}_{-\SBGminBmodelSPLGammaLowerDelta}}
\newcommand\SBGminBmodelTableSummary{SBG & \SBGminBmodel & $\SBGminBmodelBayes$ & $\SBGminBmodelSPLBayes$ & $\SBGminBmodelSPLGammaSummary$ & $\SBGminBmodelSPLPhiSummary$}
\newcommand\TavecchilowPower{Tavecchio et al.~\cite{Tavecchio:2014eia}}
\newcommand\TavecchilowPowerBayes{0.0730}
\newcommand\TavecchilowPowerSPLBayes{1.04}
\newcommand\TavecchilowPowerSPLPhiLower{2.22}
\newcommand\TavecchilowPowerSPLPhiMode{3.7}
\newcommand\TavecchilowPowerSPLPhiUpper{5.09}
\newcommand\TavecchilowPowerSPLPhiLowerDelta{\eval{\TavecchilowPowerSPLPhiMode - \TavecchilowPowerSPLPhiLower}}
\newcommand\TavecchilowPowerSPLPhiUpperDelta{\eval{\TavecchilowPowerSPLPhiUpper - \TavecchilowPowerSPLPhiMode}}
\newcommand\TavecchilowPowerSPLPhiSummary{\TavecchilowPowerSPLPhiMode^{+\TavecchilowPowerSPLPhiUpperDelta}_{-\TavecchilowPowerSPLPhiLowerDelta}}
\newcommand\TavecchilowPowerSPLGammaLower{3.39}
\newcommand\TavecchilowPowerSPLGammaMode{3.88}
\newcommand\TavecchilowPowerSPLGammaUpper{4.53}
\newcommand\TavecchilowPowerSPLGammaLowerDelta{\eval{\TavecchilowPowerSPLGammaMode - \TavecchilowPowerSPLGammaLower}}
\newcommand\TavecchilowPowerSPLGammaUpperDelta{\eval{\TavecchilowPowerSPLGammaUpper - \TavecchilowPowerSPLGammaMode}}
\newcommand\TavecchilowPowerSPLGammaSummary{\TavecchilowPowerSPLGammaMode^{+\TavecchilowPowerSPLGammaUpperDelta}_{-\TavecchilowPowerSPLGammaLowerDelta}}
\newcommand\TavecchilowPowerTableSummary{LLBLLac & \TavecchilowPower & $\TavecchilowPowerBayes$ & $\TavecchilowPowerSPLBayes$ & $\TavecchilowPowerSPLGammaSummary$ & $\TavecchilowPowerSPLPhiSummary$}
\newcommand\TDEWinterBiehl{Biehl et al.~\cite{Biehl:2017hnb}}
\newcommand\TDEWinterBiehlBayes{8.66\times10^{-7}}
\newcommand\TDEWinterBiehlSPLBayes{0.362}
\newcommand\TDEWinterBiehlSPLPhiLower{4.06}
\newcommand\TDEWinterBiehlSPLPhiMode{5.09}
\newcommand\TDEWinterBiehlSPLPhiUpper{7.16}
\newcommand\TDEWinterBiehlSPLPhiLowerDelta{\eval{\TDEWinterBiehlSPLPhiMode - \TDEWinterBiehlSPLPhiLower}}
\newcommand\TDEWinterBiehlSPLPhiUpperDelta{\eval{\TDEWinterBiehlSPLPhiUpper - \TDEWinterBiehlSPLPhiMode}}
\newcommand\TDEWinterBiehlSPLPhiSummary{\TDEWinterBiehlSPLPhiMode^{+\TDEWinterBiehlSPLPhiUpperDelta}_{-\TDEWinterBiehlSPLPhiLowerDelta}}
\newcommand\TDEWinterBiehlSPLGammaLower{2.97}
\newcommand\TDEWinterBiehlSPLGammaMode{3.35}
\newcommand\TDEWinterBiehlSPLGammaUpper{3.75}
\newcommand\TDEWinterBiehlSPLGammaLowerDelta{\eval{\TDEWinterBiehlSPLGammaMode - \TDEWinterBiehlSPLGammaLower}}
\newcommand\TDEWinterBiehlSPLGammaUpperDelta{\eval{\TDEWinterBiehlSPLGammaUpper - \TDEWinterBiehlSPLGammaMode}}
\newcommand\TDEWinterBiehlSPLGammaSummary{\TDEWinterBiehlSPLGammaMode^{+\TDEWinterBiehlSPLGammaUpperDelta}_{-\TDEWinterBiehlSPLGammaLowerDelta}}
\newcommand\TDEWinterBiehlTableSummary{GRB & \TDEWinterBiehl & $\TDEWinterBiehlBayes$ & $\TDEWinterBiehlSPLBayes$ & $\TDEWinterBiehlSPLGammaSummary$ & $\TDEWinterBiehlSPLPhiSummary$}
\newcommand\GlobusGZKGRBEvolMixCompbetaTwo{Globus et al. Evol Mix Comp Beta 2~\cite{Globus:2017zsq}}
\newcommand\GlobusGZKGRBEvolMixCompbetaTwoBayes{2.39\times10^{-10}}
\newcommand\GlobusGZKGRBEvolMixCompbetaTwoSPLBayes{1.02}
\newcommand\GlobusGZKGRBEvolMixCompbetaTwoSPLPhiLower{4.2}
\newcommand\GlobusGZKGRBEvolMixCompbetaTwoSPLPhiMode{5.46}
\newcommand\GlobusGZKGRBEvolMixCompbetaTwoSPLPhiUpper{7.28}
\newcommand\GlobusGZKGRBEvolMixCompbetaTwoSPLPhiLowerDelta{\eval{\GlobusGZKGRBEvolMixCompbetaTwoSPLPhiMode - \GlobusGZKGRBEvolMixCompbetaTwoSPLPhiLower}}
\newcommand\GlobusGZKGRBEvolMixCompbetaTwoSPLPhiUpperDelta{\eval{\GlobusGZKGRBEvolMixCompbetaTwoSPLPhiUpper - \GlobusGZKGRBEvolMixCompbetaTwoSPLPhiMode}}
\newcommand\GlobusGZKGRBEvolMixCompbetaTwoSPLPhiSummary{\GlobusGZKGRBEvolMixCompbetaTwoSPLPhiMode^{+\GlobusGZKGRBEvolMixCompbetaTwoSPLPhiUpperDelta}_{-\GlobusGZKGRBEvolMixCompbetaTwoSPLPhiLowerDelta}}
\newcommand\GlobusGZKGRBEvolMixCompbetaTwoSPLGammaLower{2.69}
\newcommand\GlobusGZKGRBEvolMixCompbetaTwoSPLGammaMode{2.95}
\newcommand\GlobusGZKGRBEvolMixCompbetaTwoSPLGammaUpper{3.14}
\newcommand\GlobusGZKGRBEvolMixCompbetaTwoSPLGammaLowerDelta{\eval{\GlobusGZKGRBEvolMixCompbetaTwoSPLGammaMode - \GlobusGZKGRBEvolMixCompbetaTwoSPLGammaLower}}
\newcommand\GlobusGZKGRBEvolMixCompbetaTwoSPLGammaUpperDelta{\eval{\GlobusGZKGRBEvolMixCompbetaTwoSPLGammaUpper - \GlobusGZKGRBEvolMixCompbetaTwoSPLGammaMode}}
\newcommand\GlobusGZKGRBEvolMixCompbetaTwoSPLGammaSummary{\GlobusGZKGRBEvolMixCompbetaTwoSPLGammaMode^{+\GlobusGZKGRBEvolMixCompbetaTwoSPLGammaUpperDelta}_{-\GlobusGZKGRBEvolMixCompbetaTwoSPLGammaLowerDelta}}
\newcommand\GlobusGZKGRBEvolMixCompbetaTwoTableSummary{UHECR & \GlobusGZKGRBEvolMixCompbetaTwo & $\GlobusGZKGRBEvolMixCompbetaTwoBayes$ & $\GlobusGZKGRBEvolMixCompbetaTwoSPLBayes$ & $\GlobusGZKGRBEvolMixCompbetaTwoSPLGammaSummary$ & $\GlobusGZKGRBEvolMixCompbetaTwoSPLPhiSummary$}
\newcommand\GlobusGZKGRBEvolMixCompbetaTwoFive{Globus et al. Evol Mix Comp Beta 2.5~\cite{Globus:2017zsq}}
\newcommand\GlobusGZKGRBEvolMixCompbetaTwoFiveBayes{4.92\times10^{-9}}
\newcommand\GlobusGZKGRBEvolMixCompbetaTwoFiveSPLBayes{1.01}
\newcommand\GlobusGZKGRBEvolMixCompbetaTwoFiveSPLPhiLower{4.18}
\newcommand\GlobusGZKGRBEvolMixCompbetaTwoFiveSPLPhiMode{5.91}
\newcommand\GlobusGZKGRBEvolMixCompbetaTwoFiveSPLPhiUpper{7.27}
\newcommand\GlobusGZKGRBEvolMixCompbetaTwoFiveSPLPhiLowerDelta{\eval{\GlobusGZKGRBEvolMixCompbetaTwoFiveSPLPhiMode - \GlobusGZKGRBEvolMixCompbetaTwoFiveSPLPhiLower}}
\newcommand\GlobusGZKGRBEvolMixCompbetaTwoFiveSPLPhiUpperDelta{\eval{\GlobusGZKGRBEvolMixCompbetaTwoFiveSPLPhiUpper - \GlobusGZKGRBEvolMixCompbetaTwoFiveSPLPhiMode}}
\newcommand\GlobusGZKGRBEvolMixCompbetaTwoFiveSPLPhiSummary{\GlobusGZKGRBEvolMixCompbetaTwoFiveSPLPhiMode^{+\GlobusGZKGRBEvolMixCompbetaTwoFiveSPLPhiUpperDelta}_{-\GlobusGZKGRBEvolMixCompbetaTwoFiveSPLPhiLowerDelta}}
\newcommand\GlobusGZKGRBEvolMixCompbetaTwoFiveSPLGammaLower{2.72}
\newcommand\GlobusGZKGRBEvolMixCompbetaTwoFiveSPLGammaMode{2.97}
\newcommand\GlobusGZKGRBEvolMixCompbetaTwoFiveSPLGammaUpper{3.23}
\newcommand\GlobusGZKGRBEvolMixCompbetaTwoFiveSPLGammaLowerDelta{\eval{\GlobusGZKGRBEvolMixCompbetaTwoFiveSPLGammaMode - \GlobusGZKGRBEvolMixCompbetaTwoFiveSPLGammaLower}}
\newcommand\GlobusGZKGRBEvolMixCompbetaTwoFiveSPLGammaUpperDelta{\eval{\GlobusGZKGRBEvolMixCompbetaTwoFiveSPLGammaUpper - \GlobusGZKGRBEvolMixCompbetaTwoFiveSPLGammaMode}}
\newcommand\GlobusGZKGRBEvolMixCompbetaTwoFiveSPLGammaSummary{\GlobusGZKGRBEvolMixCompbetaTwoFiveSPLGammaMode^{+\GlobusGZKGRBEvolMixCompbetaTwoFiveSPLGammaUpperDelta}_{-\GlobusGZKGRBEvolMixCompbetaTwoFiveSPLGammaLowerDelta}}
\newcommand\GlobusGZKGRBEvolMixCompbetaTwoFiveTableSummary{UHECR & \GlobusGZKGRBEvolMixCompbetaTwoFive & $\GlobusGZKGRBEvolMixCompbetaTwoFiveBayes$ & $\GlobusGZKGRBEvolMixCompbetaTwoFiveSPLBayes$ & $\GlobusGZKGRBEvolMixCompbetaTwoFiveSPLGammaSummary$ & $\GlobusGZKGRBEvolMixCompbetaTwoFiveSPLPhiSummary$}
\newcommand\GlobusGZKGRBEvolProton{Globus et al. Evol Proton~\cite{Globus:2017zsq}}
\newcommand\GlobusGZKGRBEvolProtonBayes{3.94\times10^{-10}}
\newcommand\GlobusGZKGRBEvolProtonSPLBayes{0.931}
\newcommand\GlobusGZKGRBEvolProtonSPLPhiLower{4.19}
\newcommand\GlobusGZKGRBEvolProtonSPLPhiMode{5.7}
\newcommand\GlobusGZKGRBEvolProtonSPLPhiUpper{7.24}
\newcommand\GlobusGZKGRBEvolProtonSPLPhiLowerDelta{\eval{\GlobusGZKGRBEvolProtonSPLPhiMode - \GlobusGZKGRBEvolProtonSPLPhiLower}}
\newcommand\GlobusGZKGRBEvolProtonSPLPhiUpperDelta{\eval{\GlobusGZKGRBEvolProtonSPLPhiUpper - \GlobusGZKGRBEvolProtonSPLPhiMode}}
\newcommand\GlobusGZKGRBEvolProtonSPLPhiSummary{\GlobusGZKGRBEvolProtonSPLPhiMode^{+\GlobusGZKGRBEvolProtonSPLPhiUpperDelta}_{-\GlobusGZKGRBEvolProtonSPLPhiLowerDelta}}
\newcommand\GlobusGZKGRBEvolProtonSPLGammaLower{2.7}
\newcommand\GlobusGZKGRBEvolProtonSPLGammaMode{2.89}
\newcommand\GlobusGZKGRBEvolProtonSPLGammaUpper{3.16}
\newcommand\GlobusGZKGRBEvolProtonSPLGammaLowerDelta{\eval{\GlobusGZKGRBEvolProtonSPLGammaMode - \GlobusGZKGRBEvolProtonSPLGammaLower}}
\newcommand\GlobusGZKGRBEvolProtonSPLGammaUpperDelta{\eval{\GlobusGZKGRBEvolProtonSPLGammaUpper - \GlobusGZKGRBEvolProtonSPLGammaMode}}
\newcommand\GlobusGZKGRBEvolProtonSPLGammaSummary{\GlobusGZKGRBEvolProtonSPLGammaMode^{+\GlobusGZKGRBEvolProtonSPLGammaUpperDelta}_{-\GlobusGZKGRBEvolProtonSPLGammaLowerDelta}}
\newcommand\GlobusGZKGRBEvolProtonTableSummary{UHECR & \GlobusGZKGRBEvolProton & $\GlobusGZKGRBEvolProtonBayes$ & $\GlobusGZKGRBEvolProtonSPLBayes$ & $\GlobusGZKGRBEvolProtonSPLGammaSummary$ & $\GlobusGZKGRBEvolProtonSPLPhiSummary$}
\newcommand\GlobusGZKNoEvolMixCompbetaTwoFive{Globus et al. No Evol Mix Comp Beta 2.5~\cite{Globus:2017zsq}}
\newcommand\GlobusGZKNoEvolMixCompbetaTwoFiveBayes{1.08\times10^{-10}}
\newcommand\GlobusGZKNoEvolMixCompbetaTwoFiveSPLBayes{1.03}
\newcommand\GlobusGZKNoEvolMixCompbetaTwoFiveSPLPhiLower{4.28}
\newcommand\GlobusGZKNoEvolMixCompbetaTwoFiveSPLPhiMode{6.0}
\newcommand\GlobusGZKNoEvolMixCompbetaTwoFiveSPLPhiUpper{7.34}
\newcommand\GlobusGZKNoEvolMixCompbetaTwoFiveSPLPhiLowerDelta{\eval{\GlobusGZKNoEvolMixCompbetaTwoFiveSPLPhiMode - \GlobusGZKNoEvolMixCompbetaTwoFiveSPLPhiLower}}
\newcommand\GlobusGZKNoEvolMixCompbetaTwoFiveSPLPhiUpperDelta{\eval{\GlobusGZKNoEvolMixCompbetaTwoFiveSPLPhiUpper - \GlobusGZKNoEvolMixCompbetaTwoFiveSPLPhiMode}}
\newcommand\GlobusGZKNoEvolMixCompbetaTwoFiveSPLPhiSummary{\GlobusGZKNoEvolMixCompbetaTwoFiveSPLPhiMode^{+\GlobusGZKNoEvolMixCompbetaTwoFiveSPLPhiUpperDelta}_{-\GlobusGZKNoEvolMixCompbetaTwoFiveSPLPhiLowerDelta}}
\newcommand\GlobusGZKNoEvolMixCompbetaTwoFiveSPLGammaLower{2.7}
\newcommand\GlobusGZKNoEvolMixCompbetaTwoFiveSPLGammaMode{2.89}
\newcommand\GlobusGZKNoEvolMixCompbetaTwoFiveSPLGammaUpper{3.15}
\newcommand\GlobusGZKNoEvolMixCompbetaTwoFiveSPLGammaLowerDelta{\eval{\GlobusGZKNoEvolMixCompbetaTwoFiveSPLGammaMode - \GlobusGZKNoEvolMixCompbetaTwoFiveSPLGammaLower}}
\newcommand\GlobusGZKNoEvolMixCompbetaTwoFiveSPLGammaUpperDelta{\eval{\GlobusGZKNoEvolMixCompbetaTwoFiveSPLGammaUpper - \GlobusGZKNoEvolMixCompbetaTwoFiveSPLGammaMode}}
\newcommand\GlobusGZKNoEvolMixCompbetaTwoFiveSPLGammaSummary{\GlobusGZKNoEvolMixCompbetaTwoFiveSPLGammaMode^{+\GlobusGZKNoEvolMixCompbetaTwoFiveSPLGammaUpperDelta}_{-\GlobusGZKNoEvolMixCompbetaTwoFiveSPLGammaLowerDelta}}
\newcommand\GlobusGZKNoEvolMixCompbetaTwoFiveTableSummary{UHECR & \GlobusGZKNoEvolMixCompbetaTwoFive & $\GlobusGZKNoEvolMixCompbetaTwoFiveBayes$ & $\GlobusGZKNoEvolMixCompbetaTwoFiveSPLBayes$ & $\GlobusGZKNoEvolMixCompbetaTwoFiveSPLGammaSummary$ & $\GlobusGZKNoEvolMixCompbetaTwoFiveSPLPhiSummary$}
\newcommand\GlobusGZKNoEvolMixCompbetaTwo{Globus et al. No Evol Mix Comp Beta 2~\cite{Globus:2017zsq}}
\newcommand\GlobusGZKNoEvolMixCompbetaTwoBayes{2.57\times10^{-11}}
\newcommand\GlobusGZKNoEvolMixCompbetaTwoSPLBayes{1.48}
\newcommand\GlobusGZKNoEvolMixCompbetaTwoSPLPhiLower{4.18}
\newcommand\GlobusGZKNoEvolMixCompbetaTwoSPLPhiMode{5.6}
\newcommand\GlobusGZKNoEvolMixCompbetaTwoSPLPhiUpper{7.21}
\newcommand\GlobusGZKNoEvolMixCompbetaTwoSPLPhiLowerDelta{\eval{\GlobusGZKNoEvolMixCompbetaTwoSPLPhiMode - \GlobusGZKNoEvolMixCompbetaTwoSPLPhiLower}}
\newcommand\GlobusGZKNoEvolMixCompbetaTwoSPLPhiUpperDelta{\eval{\GlobusGZKNoEvolMixCompbetaTwoSPLPhiUpper - \GlobusGZKNoEvolMixCompbetaTwoSPLPhiMode}}
\newcommand\GlobusGZKNoEvolMixCompbetaTwoSPLPhiSummary{\GlobusGZKNoEvolMixCompbetaTwoSPLPhiMode^{+\GlobusGZKNoEvolMixCompbetaTwoSPLPhiUpperDelta}_{-\GlobusGZKNoEvolMixCompbetaTwoSPLPhiLowerDelta}}
\newcommand\GlobusGZKNoEvolMixCompbetaTwoSPLGammaLower{2.69}
\newcommand\GlobusGZKNoEvolMixCompbetaTwoSPLGammaMode{2.92}
\newcommand\GlobusGZKNoEvolMixCompbetaTwoSPLGammaUpper{3.12}
\newcommand\GlobusGZKNoEvolMixCompbetaTwoSPLGammaLowerDelta{\eval{\GlobusGZKNoEvolMixCompbetaTwoSPLGammaMode - \GlobusGZKNoEvolMixCompbetaTwoSPLGammaLower}}
\newcommand\GlobusGZKNoEvolMixCompbetaTwoSPLGammaUpperDelta{\eval{\GlobusGZKNoEvolMixCompbetaTwoSPLGammaUpper - \GlobusGZKNoEvolMixCompbetaTwoSPLGammaMode}}
\newcommand\GlobusGZKNoEvolMixCompbetaTwoSPLGammaSummary{\GlobusGZKNoEvolMixCompbetaTwoSPLGammaMode^{+\GlobusGZKNoEvolMixCompbetaTwoSPLGammaUpperDelta}_{-\GlobusGZKNoEvolMixCompbetaTwoSPLGammaLowerDelta}}
\newcommand\GlobusGZKNoEvolMixCompbetaTwoTableSummary{UHERC & \GlobusGZKNoEvolMixCompbetaTwo & $\GlobusGZKNoEvolMixCompbetaTwoBayes$ & $\GlobusGZKNoEvolMixCompbetaTwoSPLBayes$ & $\GlobusGZKNoEvolMixCompbetaTwoSPLGammaSummary$ & $\GlobusGZKNoEvolMixCompbetaTwoSPLPhiSummary$}
\newcommand\GlobusGZKNoEvolProton{Globus et al. No Evol Proton~\cite{Globus:2017zsq}}
\newcommand\GlobusGZKNoEvolProtonBayes{3.17\times10^{-11}}
\newcommand\GlobusGZKNoEvolProtonSPLBayes{1.05}
\newcommand\GlobusGZKNoEvolProtonSPLPhiLower{4.27}
\newcommand\GlobusGZKNoEvolProtonSPLPhiMode{5.75}
\newcommand\GlobusGZKNoEvolProtonSPLPhiUpper{7.26}
\newcommand\GlobusGZKNoEvolProtonSPLPhiLowerDelta{\eval{\GlobusGZKNoEvolProtonSPLPhiMode - \GlobusGZKNoEvolProtonSPLPhiLower}}
\newcommand\GlobusGZKNoEvolProtonSPLPhiUpperDelta{\eval{\GlobusGZKNoEvolProtonSPLPhiUpper - \GlobusGZKNoEvolProtonSPLPhiMode}}
\newcommand\GlobusGZKNoEvolProtonSPLPhiSummary{\GlobusGZKNoEvolProtonSPLPhiMode^{+\GlobusGZKNoEvolProtonSPLPhiUpperDelta}_{-\GlobusGZKNoEvolProtonSPLPhiLowerDelta}}
\newcommand\GlobusGZKNoEvolProtonSPLGammaLower{2.69}
\newcommand\GlobusGZKNoEvolProtonSPLGammaMode{2.89}
\newcommand\GlobusGZKNoEvolProtonSPLGammaUpper{3.13}
\newcommand\GlobusGZKNoEvolProtonSPLGammaLowerDelta{\eval{\GlobusGZKNoEvolProtonSPLGammaMode - \GlobusGZKNoEvolProtonSPLGammaLower}}
\newcommand\GlobusGZKNoEvolProtonSPLGammaUpperDelta{\eval{\GlobusGZKNoEvolProtonSPLGammaUpper - \GlobusGZKNoEvolProtonSPLGammaMode}}
\newcommand\GlobusGZKNoEvolProtonSPLGammaSummary{\GlobusGZKNoEvolProtonSPLGammaMode^{+\GlobusGZKNoEvolProtonSPLGammaUpperDelta}_{-\GlobusGZKNoEvolProtonSPLGammaLowerDelta}}
\newcommand\GlobusGZKNoEvolProtonTableSummary{UHECR & \GlobusGZKNoEvolProton & $\GlobusGZKNoEvolProtonBayes$ & $\GlobusGZKNoEvolProtonSPLBayes$ & $\GlobusGZKNoEvolProtonSPLGammaSummary$ & $\GlobusGZKNoEvolProtonSPLPhiSummary$}
% segmented power-law information
\newcommand\segmentEUno{4.20\cdot 10^4}
\newcommand\segmentEDos{8.83\cdot 10^4}
\newcommand\segmentETres{1.86\cdot 10^5}
\newcommand\segmentECuatro{3.91\cdot 10^5}
\newcommand\segmentECinco{8.23\cdot 10^5}
\newcommand\segmentESeis{1.73\cdot 10^6}
\newcommand\segmentESiete{3.64\cdot 10^6}
\newcommand\segmentEOcho{7.67\cdot 10^7}
\usepackage{tikz}
\usepackage{environ}

\usetikzlibrary{calc,trees,positioning,arrows,chains,shapes.geometric,%
    decorations.pathreplacing,decorations.pathmorphing,shapes,%
    matrix,shapes.symbols,backgrounds,fit} % required in the preamble
\usepackage{varwidth}

\makeatletter
\newsavebox{\measure@tikzpicture}
\NewEnviron{scaletikzpicturetowidth}[1]{%
  \def\tikz@width{#1}%
  \def\tikzscale{1}\begin{lrbox}{\measure@tikzpicture}%
  \BODY
  \end{lrbox}%
  \pgfmathparse{#1/\wd\measure@tikzpicture}%
  \edef\tikzscale{\pgfmathresult}%
  \BODY
}
\makeatother

\tikzset{
	%Define standard arrow tip
	>=stealth',
	%Define style for boxes
	box/.style={
		rectangle,
		rounded corners,
		dashed,
		draw=black, very thick,
		minimum height=2em,
		text centered,
		execute at begin node={\begin{varwidth}{28em}},
		execute at end node={\end{varwidth}}},
	solidbox/.style={
		rectangle,
		rounded corners,
		draw=black, very thick,
		minimum height=2em,
		text centered,
		execute at begin node={\begin{varwidth}{28em}},
			execute at end node={\end{varwidth}}},
    bigsolidbox/.style={
		rectangle,
		rounded corners,
		draw=black, very thick,
		minimum height=6cm,
		text centered,
		execute at begin node={\begin{varwidth}{28em}},
			execute at end node={\end{varwidth}}},
	% Define arrow style
	fw_arrow/.style={
		->,
		thick,
		shorten <=2pt,
		shorten >=2pt,},
	bw_arrow/.style={
		<-,
		thick,
		shorten <=2pt,
		shorten >=2pt,}
%	bigbox/.style={blue!50, thick, fill=blue!10, rounded corners, rectangle}
}

\definecolor{mc_gen_color}{RGB}{250,138,31}

\definecolor{det_sim_color}{RGB}{227,66,55}

\definecolor{llh_color}{RGB}{128,0,128}

\tikzstyle{bigboxGeneration} = [draw=mc_gen_color!50, thick, fill=mc_gen_color!20, rounded corners, rectangle]

\tikzstyle{bigboxDetector} = [draw=det_sim_color!50, thick, fill=det_sim_color!20, rounded corners, rectangle]

\tikzstyle{bigboxAnalysis} = [draw=llh_color!50, thick, fill=llh_color!20, rounded corners, rectangle]
   
\newcommand{\cmark}{\text{\ding{51}}}
\newcommand{\xmark}{\text{\ding{55}}}

\settitle{Precision measurements of the astrophysical neutrino flux}
\setauthor{Austin Schneider}
\setdepartment{Physics}
\doctors
\setgraddate{2020}
\setdefensedate{Some date in the near future}

\setfoca{Albrecht Karle}{Professor}{Physics}

\setabstract{The IceCube neutrino observatory has established the existence of a high-energy isotropic neutrino flux of astrophysical origin. This discovery was made using events interacting within a fiducial region of the detector surrounded by an active veto with reconstructed energy above $\SI{60}\TeV$, commonly known as the high-energy starting event sample or HESE. We perform a complete revisit of the HESE sample with an additional $\SI{4.5}\year$ of data, newer glacial ice models, and improved systematics treatment. This paper gives a detailed description of the sample, reports on the latest astrophysical neutrino flux measurements, and presents a source search for astrophysical neutrinos. We give the compatibility of these observations with detailed isotropic flux models proposed in the literature as well as generic power-law-like scenarios. We find that the astrophysical spectrum, when assumed equal for neutrinos and anti-neutrinos and among neutrino flavors, is compatible with an unbroken power law, with a preferred spectral index of $2.88^{+0.20}_{-0.19}$ for the $\SI{68.3}\percent$ confidence interval.}
